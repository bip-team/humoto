\chapter{Reference}

This chapter contains various derivations and definitions which are used by
multiple modules of \projectname.


%%%%%%%%%%%%%%%%%%%%%%%%%%%%%%%%%%%%%%%%%%%%%%%%%%%%%%%%%%%%%%%%%%%%%%%%%%%%%%%%%%%%%%%%%%%%%%%%%%%%%%
%%%%%%%%%%%%%%%%%%%%%%%%%%%%%%%%%%%%%%%%%%%%%%%%%%%%%%%%%%%%%%%%%%%%%%%%%%%%%%%%%%%%%%%%%%%%%%%%%%%%%%
%%%%%%%%%%%%%%%%%%%%%%%%%%%%%%%%%%%%%%%%%%%%%%%%%%%%%%%%%%%%%%%%%%%%%%%%%%%%%%%%%%%%%%%%%%%%%%%%%%%%%%
\section{Model Predictive Control}\label{sec.condensing}

\acf{MPC} is a branch of control theory, where the control inputs are generated
by optimizing behavior of a system over certain preview horizon.

A linear model of the system has the following form:
%
\begin{equation}
        \V{x}_{k+1} = \M{A}_k \V{x}_k + \M{B}_k \V{u}_k, \quad k = 0, \dots, N
\end{equation}
%
where $\V{x}_k$ and $\V{u}_k$ are $k$-th state and control input respectively,
while $N$ is the length of preview (prediction) horizon.

Ouptput of the system can be defined in different ways, here we assume that it
depends on the preceding state and control
%
\begin{equation}
        \V{y}_{k+1} = \M{D}_k \V{x}_k + \M{E}_k \V{u}_k, \quad k = 0, \dots, N
\end{equation}
%


%%%%%%%%%%%%%%%%%%%%%%%%%%%%%%%%%%%%%%%%%%%%%%%%%%%%%%%%%%%%%%%%%%%%%%%%%%%%%%%%%%%%%%%%%%%%%%%%%%%%%%
\subsection{Condensing}\label{sec.condensing}

Condensing amounts to finding such matrices $\M{U}_{x}$ and $\M{U}_{u}$ that
%
\begin{equation}
    \V{v}_{x} = \M{U}_{x} \V{x}_0 + \M{U}_{u} \V{v}_{u},
\end{equation}
%
where
%
\begin{equation}
\begin{split}
    \V{v}_{x} &= (\V{x}_1, \dots, \V{x}_N), \\
    \V{v}_{u} &= (\V{u}_0, \dots, \V{u}_{N-1}).
\end{split}
\end{equation}
%

Similarly for the output
\begin{equation}
    \V{v}_{y} = \M{O}_{x} \V{x}_0 + \M{O}_{u} \V{v}_{u}.
\end{equation}
Note that
\begin{equation}
    \M{O}_{x}
    =
    \begin{bmatrix}
        \diag{k=0 \dots N-1}{D_k}  &  \M{0} \\
    \end{bmatrix}
    \begin{bmatrix}
        \M{0} \\
        \M{U}_{x} \\
    \end{bmatrix}
    \quad
    \quad
    \M{O}_{u}
    =
    \begin{bmatrix}
        \diag{k=0 \dots N-1}{D_k}  &  \M{0} \\
    \end{bmatrix}
    \begin{bmatrix}
        \M{0} \\
        \M{U}_{u} \\
    \end{bmatrix}
    +
    \diag{k=0 \dots N-1}{E_k}
    \M{U}_{u}
\end{equation}


%%%%%%%%%%%%%%%%%%%%%%%%%%%%%%%%%%%%%%%%%%%%%%%%%%%%%%%%%%%%%%%%%%%%%%%%%%%%%%%%%%%%%%%%%%%%%%%%%%%%%%
\subsubsection{Time-variant system}\label{sec.condensing_gen}
\begin{equation}
    \M{U}_{x} =
        \begin{bmatrix}
        \M{A}_0    \\
        \M{A}_1 \M{A}_0  \\
        \vdots           \\
        \M{A}_{N-1} \dots \M{A}_0 \\
        \end{bmatrix}
    \quad\quad
    \M{U}_{u} =
        \begin{bmatrix}
        \M{B}_0                             & \M{0}                                 & \dots & \M{0} \\
        \M{A}_1 \M{B}_0                     & \M{B}_1                               & \dots & \M{0} \\
        \vdots                              & \vdots                                & \ddots& \vdots \\
        \M{A}_{N-1} \dots \M{A}_1 \M{B}_0   & \M{A}_{N-1} \dots \M{A}_2 \M{B}_1     & \dots & \M{B}_{N-1} \\
        \end{bmatrix}
\end{equation}

\begin{equation}
    \M{O}_{x} =
        \begin{bmatrix}
        \M{D}_0    \\
        \M{D}_1 \M{A}_0    \\
        \M{D}_2 \M{A}_1 \M{A}_0  \\
        \vdots           \\
        \M{D}_{N-1} \M{A}_{N-2} \dots \M{A}_0 \\
        \end{bmatrix}
    \quad\quad
    \M{O}_{u} =
        \begin{bmatrix}
        \M{E}_0                             & \M{0}                                 & \dots & \M{0} \\
        \M{D}_1 \M{B}_0                     & \M{E}_1                               & \dots & \M{0} \\
        \M{D}_2 \M{A}_1 \M{B}_0             & \M{D}_2 \M{B}_1                       & \dots & \M{0} \\
        \vdots                              & \vdots                                & \ddots& \vdots \\
        \M{D}_{N-1} \M{A}_{N-2} \dots \M{A}_1 \M{B}_0   & \M{D}_{N-1} \M{A}_{N-2} \dots \M{A}_2 \M{B}_1     & \dots & \M{E}_{N-1} \\
        \end{bmatrix}
\end{equation}


%%%%%%%%%%%%%%%%%%%%%%%%%%%%%%%%%%%%%%%%%%%%%%%%%%%%%%%%%%%%%%%%%%%%%%%%%%%%%%%%%%%%%%%%%%%%%%%%%%%%%%
\subsubsection{System is varying in the first preview interval}
\begin{equation}
    \M{U}_{x} =
        \begin{bmatrix}
        \M{A}_0    \\
        \M{A} \M{A}_0  \\
        \vdots   \\
        \M{A}^{N-1} \M{A}_0 \\
        \end{bmatrix}
    \quad\quad
    \M{U}_{u} =
        \begin{bmatrix}
        \M{B}_0                 & \M{0}                 & \dots & \M{0} \\
        \M{A} \M{B}_0           & \M{B}                 & \dots & \M{0} \\
        \vdots                  & \vdots                & \ddots& \vdots \\
        \M{A}^{N-1} \M{B}_0     & \M{A}^{N-2} \M{B}     & \dots & \M{B} \\
        \end{bmatrix}
\end{equation}

\begin{equation}
    \M{O}_{x} =
        \begin{bmatrix}
        \M{D}_0    \\
        \M{D} \M{A}_0    \\
        \M{D} \M{A} \M{A}_0  \\
        \vdots           \\
        \M{D} \M{A}^{N-2} \M{A}_0 \\
        \end{bmatrix}
    \quad\quad
    \M{O}_{u} =
        \begin{bmatrix}
        \M{E}_0                             & \M{0}                                 & \dots & \M{0} \\
        \M{D} \M{B}_0                       & \M{E}                                 & \dots & \M{0} \\
        \M{D} \M{A} \M{B}_0                 & \M{D} \M{B}                           & \dots & \M{0} \\
        \vdots                              & \vdots                                & \ddots& \vdots \\
        \M{D} \M{A}^{N-2} \M{B}_0           & \M{D} \M{A}^{N-3} \M{B}               & \dots & \M{E} \\
        \end{bmatrix}
\end{equation}

%%%%%%%%%%%%%%%%%%%%%%%%%%%%%%%%%%%%%%%%%%%%%%%%%%%%%%%%%%%%%%%%%%%%%%%%%%%%%%%%%%%%%%%%%%%%%%%%%%%%%%
\subsubsection{Time-invariant system}
\begin{equation}
    \M{U}_{x} =
        \begin{bmatrix}
        \M{A}    \\
        \M{A}^2  \\
        \vdots   \\
        \M{A}^N  \\
        \end{bmatrix}
    \quad\quad
    \M{U}_{u} =
        \begin{bmatrix}
        \M{B}                   & \M{0}                 & \dots & \M{0} \\
        \M{A} \M{B}             & \M{B}                 & \dots & \M{0} \\
        \vdots                  & \vdots                & \ddots& \vdots \\
        \M{A}^{N-1} \M{B}       & \M{A}^{N-2} \M{B}     & \dots & \M{B} \\
        \end{bmatrix}
\end{equation}


\begin{equation}
    \M{O}_{x} =
        \begin{bmatrix}
        \M{D}    \\
        \M{D} \M{A}    \\
        \M{D} \M{A}^2  \\
        \vdots           \\
        \M{D} \M{A}^{N-1} \\
        \end{bmatrix}
    \quad\quad
    \M{O}_{u} =
        \begin{bmatrix}
        \M{E}                               & \M{0}                                 & \dots & \M{0} \\
        \M{D} \M{B}                         & \M{E}                                 & \dots & \M{0} \\
        \M{D} \M{A} \M{B}                   & \M{D} \M{B}                           & \dots & \M{0} \\
        \vdots                              & \vdots                                & \ddots& \vdots \\
        \M{D} \M{A}^{N-2} \M{B}             & \M{D} \M{A}^{N-3} \M{B}               & \dots & \M{E} \\
        \end{bmatrix}
\end{equation}



%%%%%%%%%%%%%%%%%%%%%%%%%%%%%%%%%%%%%%%%%%%%%%%%%%%%%%%%%%%%%%%%%%%%%%%%%%%%%%%%
%%%%%%%%%%%%%%%%%%%%%%%%%%%%%%%%%%%%%%%%%%%%%%%%%%%%%%%%%%%%%%%%%%%%%%%%%%%%%%%%
%%%%%%%%%%%%%%%%%%%%%%%%%%%%%%%%%%%%%%%%%%%%%%%%%%%%%%%%%%%%%%%%%%%%%%%%%%%%%%%%
\section{Triple integrator (discrete-time)}

In some cases it is convenient to represent model of the system with triple
integrator
%
\begin{equation}
    \V{x}_{k+1}
    =
    \M{A}_k
    \V{x}_{k}
    +
    \M{B}_k
    \dddot{x}_{k},
    \quad
    \V{x}_{k}
    =
    (
        x_k,
        \dot{x}_k,
        \ddot{x}_k
    )
\end{equation}
%
%
\begin{equation}
    \M{A}_k
    =
    \begin{bmatrix}
        1 & T_k & \frac{T_k^2}{2}\\
        0 & 1 & T_k\\
        0 & 0 & 1
    \end{bmatrix}
    ,
    \quad
    \M{B}_k
    =
    \begin{bmatrix}
        \frac{T_k^3}{6} \\
        \frac{T_k^2}{2} \\
        T_k
    \end{bmatrix}
\end{equation}
%


The corresponding matrices are defined in \sn{Maxima} as
%
\begin{listingtcb}{Maxima}
\begin{deflisting}
A: matrix([1, T_k, T_k^2/2], [0, 1, T_k], [0, 0, 1]);
B: matrix([T_k^3/6], [T_k^2/2], [T_k]);
X0: matrix([x0],[dx0],[ddx0]);
U: matrix([dddx0]);
X1: matrix([x1],[dx1],[ddx1]);
As: matrix([1, T_s, T_s^2/2], [0, 1, T_s], [0, 0, 1]);
Bs: matrix([T_s^3/6], [T_s^2/2], [T_s]);
\end{deflisting}
\end{listingtcb}
%


It is possible to reformulate the triple integrator model in order to use
acceleration, velocity, or position from the next state as control input
instead of the jerk. This may be useful in the cases when constraints on the
state variables can be represented with simple bounds. The adjusted model
produces the same motions, but numerical properties of its matrices may be
different.


%%%%%%%%%%%%%%%%%%%%%%%%%%%%%%%%%%%%%%%%%%%%%%%%%%%%%%%%%%%%%%%%%%%%%%%%%%%%%%%%
\subsection{Controlled using acceleration}

\begin{align}
    \V{x}_{k+1}
    &=
    \M{A}_k
    \V{x}_{k}
    +
    \M{B}_k
    \ddot{x}_{k+1}
    \\
    \dddot{x}_{k}
    &=
    \M{D}_k
    \V{x}_{k}
    +
    \M{E}_k
    \ddot{x}_{k+1}
\end{align}


%
\begin{equation}
    \M{A}_k =
        \begin{bmatrix}
            1   & T_k   & \frac{T_k^2}{3} \\
            0   & 1     & \frac{T_k}{2} \\
            0   & 0     & 0 \\
        \end{bmatrix},
    \quad
    \M{B}_k =
        \begin{bmatrix}
            \frac{T_k^2}{6} \\
            \frac{T_k}{2} \\
            1 \\
        \end{bmatrix},
    \quad
    \M{D}_k =
        \begin{bmatrix}
            0 & 0 & - \frac{1}{T_k}
        \end{bmatrix},
    \quad
    \M{E}_k =
        \begin{bmatrix}
            \frac{1}{T_k} \\
        \end{bmatrix}
\end{equation}
%

For $T_s \in [0, T_k]$
%
\begin{equation}
    \M{A}_{s,k} =
        \begin{bmatrix}
            1 & {T_s}   &  - \frac{{T_s}^3 - 3 {T_k} {T_s}^2}{6 {T_k}} \\
            0 & 1       &  - \frac{{T_s}^2 - 2 {T_k} {T_s}}{2 {T_k}} \\
            0 & 0       &  - \frac{{T_s} - {T_k}}{{T_k}} \\
        \end{bmatrix}
    ,
    \quad
    \M{B}_{s,k} =
        \begin{bmatrix}
            \frac{{T_s}^3}{6 {T_k}} \\
            \frac{{T_s}^2}{2 {T_k}} \\
            \frac{{T_s}}{{T_k}} \\
        \end{bmatrix}
\end{equation}
%


\begin{listingtcb}{Maxima}
\begin{deflisting}
e: solve([(A.X0 + B.U)[3][1] = (X1)[3][1]], [dddx0]);
Dnew: coefmatrix([rhs(e[1])], list_matrix_entries(X0));
Enew: coefmatrix([rhs(e[1])], [ddx1]);
e: subst(e, A.X0 + B.U);
Anew: coefmatrix(list_matrix_entries(e), list_matrix_entries(X0));
Bnew: coefmatrix(list_matrix_entries(e), [ddx1]);
tex(Anew);
tex(Bnew);
tex(Dnew);
tex(Enew);
/* subsampling */
e: solve([(A.X0 + B.U)[3][1] = (X1)[3][1]], [dddx0]);
e: subst(e, As.X0 + Bs.U);
Asnew: coefmatrix(list_matrix_entries(e), list_matrix_entries(X0));
Bsnew: coefmatrix(list_matrix_entries(e), [ddx1]);
tex(Asnew);
tex(Bsnew);
/* check */
subst([T_s = T_k], Asnew) - Anew;
subst([T_s = T_k], Bsnew) - Bnew;
\end{deflisting}
\end{listingtcb}



%%%%%%%%%%%%%%%%%%%%%%%%%%%%%%%%%%%%%%%%%%%%%%%%%%%%%%%%%%%%%%%%%%%%%%%%%%%%%%%%
\subsection{Controlled using velocity}

%
\begin{align}
    \V{x}_{k+1}
    &=
    \M{A}_k
    \V{x}_{k}
    +
    \M{B}_k
    \dot{x}_{k+1}
    \\
    \dddot{x}_{k}
    &=
    \M{D}_k
    \V{x}_{k}
    +
    \M{E}_k
    \dot{x}_{k+1}
\end{align}
%

%
\begin{equation}
    \M{A}_k =
        \begin{bmatrix}
            1     & \frac{2 T_k}{3}     & \frac{T_k^2}{6} \\
            0     & 0                   & 0 \\
            0     & -\frac{2}{T_k}      &  - 1 \\
        \end{bmatrix},
    \quad
    \M{B}_k =
        \begin{bmatrix}
            \frac{T_k}{3} \\
            1 \\
            \frac{2}{T_k} \\
        \end{bmatrix},
    \quad
    \M{D}_k =
        \begin{bmatrix}
            0 & -\frac{2}{T_k^2} & -\frac{2}{T_k} \\
        \end{bmatrix},
    \quad
    \M{E}_k =
        \begin{bmatrix}
            \frac{2}{T_k^2} \\
        \end{bmatrix}
\end{equation}
%


For $T_s \in [0, T_k]$
%
\begin{equation}
    \M{A}_{s,k} =
        \begin{bmatrix}
            1 &  - \frac{{T_s}^3 - 3  {T_k}^2 {T_s}}{3 {T_k}^2} &  - \frac{2 {T_s}^3 - 3 {T_k} {T_s}^2}{6 {T_k}} \\
            0 &  - \frac{{T_s}^2 - {T_k}^2}{{T_k}^2}            &  - \frac{{T_s}^2 - {T_k} {T_s}}{{T_k}} \\
            0 &  - \frac{2 {T_s}}{{T_k}^2}                      &  - \frac{2 {T_s} - {T_k}}{{T_k}} \\
        \end{bmatrix}
    ,
    \quad
    \M{B}_{s,k} =
        \begin{bmatrix}
            \frac{{T_s}^3}{3 {T_k}^2} \\
            \frac{{T_s}^2}{{T_k}^2} \\
            \frac{2 {T_s}}{{T_k}^2} \\
        \end{bmatrix}
\end{equation}


\begin{listingtcb}{Maxima}
\begin{deflisting}
e: solve([(A.X0 + B.U)[2][1] = (X1)[2][1]], [dddx0]);
Dnew: coefmatrix([rhs(e[1])], list_matrix_entries(X0));
Enew: coefmatrix([rhs(e[1])], [dx1]);
e: subst(e, A.X0 + B.U);
Anew: coefmatrix(list_matrix_entries(e), list_matrix_entries(X0));
Bnew: coefmatrix(list_matrix_entries(e), [dx1]);
tex(Anew);
tex(Bnew);
tex(Dnew);
tex(Enew);
/* subsampling */
e: solve([(A.X0 + B.U)[2][1] = (X1)[2][1]], [dddx0]);
e: subst(e, As.X0 + Bs.U);
Asnew: coefmatrix(list_matrix_entries(e), list_matrix_entries(X0));
Bsnew: coefmatrix(list_matrix_entries(e), [dx1]);
tex(Asnew);
tex(Bsnew);
/* check */
subst([T_s = T_k], Asnew) - Anew;
subst([T_s = T_k], Bsnew) - Bnew;
\end{deflisting}
\end{listingtcb}



%%%%%%%%%%%%%%%%%%%%%%%%%%%%%%%%%%%%%%%%%%%%%%%%%%%%%%%%%%%%%%%%%%%%%%%%%%%%%%%%
\subsection{Controlled using position}

\begin{align}
    \V{x}_{k+1}
    &=
    \M{A}_k
    \V{x}_{k}
    +
    \M{B}_k
    {x}_{k+1}
    \\
    \dddot{x}_{k}
    &=
    \M{D}_k
    \V{x}_{k}
    +
    \M{E}_k
    {x}_{k+1}
\end{align}


%
\begin{equation}
    \M{A}_k =
        \begin{bmatrix}
            0                   &   0               & 0 \\
            - \frac{3}{T_k}     &  - 2              &  - \frac{T_k}{2} \\
            - \frac{6}{T_k^2}   &  - \frac{6}{T_k}  &  - 2 \\
        \end{bmatrix},
    \quad
    \M{B}_k =
        \begin{bmatrix}
            1 \\
            \frac{3}{T_k} \\
            \frac{6}{T_k^2} \\
        \end{bmatrix},
    \quad
    \M{D}_k =
        \begin{bmatrix}
            -\frac{6}{T_k^3} &   -\frac{6}{T_k^2} & -\frac{3}{T_k} \\
        \end{bmatrix},
    \quad
    \M{E}_k =
        \begin{bmatrix}
            \frac{6}{T_k^3} \\
        \end{bmatrix}
\end{equation}
%


For $T_s \in [0, T_k]$
%
\begin{equation}
    \M{A}_{s,k} =
        \begin{bmatrix}
            - \frac{{T_s}^3 - {T_k}^3}{{T_k}^3} &  - \frac{{T_s}^3 - {T_k}^2 {T_s}}{{T_k}^2}    &  - \frac{{T_s}^3 - {T_k}{T_s}^2}{2 {T_k}} \\
            - \frac{3 {T_s}^2}{{T_k}^3}         &  - \frac{3 {T_s}^2 - {T_k}^2}{{T_k}^2}        &  -  \frac{3 {T_s}^2 - 2 {T_k} {T_s}}{2 {T_k}} \\
            - \frac{6 {T_s}}{{T_k}^3}           &  - \frac{6 {T_s}}{{T_k}^2}                    &  - \frac{3 {T_s} - {T_k}}{{T_k}} \\
        \end{bmatrix}
    ,
    \quad
    \M{B}_{s,k} =
        \begin{bmatrix}
            \frac{{T_s}^3}{{T_k}^3} \\
            \frac{3 {T_s}^2}{{T_k}^3} \\
            \frac{6 {T_s}}{{T_k}^3} \\
        \end{bmatrix}
\end{equation}
%

\begin{listingtcb}{Maxima}
\begin{deflisting}
e: solve([(A.X0 + B.U)[1][1] = (X1)[1][1]], [dddx0]);
Dnew: coefmatrix([rhs(e[1])], list_matrix_entries(X0));
Enew: coefmatrix([rhs(e[1])], [x1]);
e: subst(e, A.X0 + B.U);
Anew: coefmatrix(list_matrix_entries(e), list_matrix_entries(X0));
Bnew: coefmatrix(list_matrix_entries(e), [x1]);
tex(Anew);
tex(Bnew);
tex(Dnew);
tex(Enew);
/* subsampling */
e: solve([(A.X0 + B.U)[1][1] = (X1)[1][1]], [dddx0]);
e: subst(e, As.X0 + Bs.U);
Asnew: coefmatrix(list_matrix_entries(e), list_matrix_entries(X0));
Bsnew: coefmatrix(list_matrix_entries(e), [x1]);
tex(Asnew);
tex(Bsnew);
/* check */
subst([T_s = T_k], Asnew) - Anew;
subst([T_s = T_k], Bsnew) - Bnew;
\end{deflisting}
\end{listingtcb}



%%%%%%%%%%%%%%%%%%%%%%%%%%%%%%%%%%%%%%%%%%%%%%%%%%%%%%%%%%%%%%%%%%%%%%%%%%%%%%%%
%%%%%%%%%%%%%%%%%%%%%%%%%%%%%%%%%%%%%%%%%%%%%%%%%%%%%%%%%%%%%%%%%%%%%%%%%%%%%%%%
%%%%%%%%%%%%%%%%%%%%%%%%%%%%%%%%%%%%%%%%%%%%%%%%%%%%%%%%%%%%%%%%%%%%%%%%%%%%%%%%
\section{Kronecker and Block Kronecker products}

The standard Kronecker product is defined as follows:
%
\begin{equation}
    \M{A} \kron \M{B}
    =
    \begin{bmatrix}
        A_{1,1} \M{B} & \dots  & A_{1,m} \M{B} \\
        \vdots        & \vdots & \vdots \\
        A_{n,1} \M{B} & \dots  & A_{n,m} \M{B} \\
    \end{bmatrix},
\end{equation}
%
where $\M{A} \in \RR^{n \times m}$ and $A_{i,j}$ denotes a scalar standing in
the $i$-th row and $j$-th column.


Derivations presented in this manual can often be expressed using a slightly
different operator:
%
\begin{equation}
    \M{A} \bkron \M{B}
    =
    \begin{bmatrix}
        \M{A} \kron \M{B}_{[1,1]}   & \dots  & \M{A} \kron \M{B}_{[1,m]} \\
        \vdots                      & \vdots & \vdots \\
        \M{A} \kron \M{B}_{[n,1]}   & \dots  & \M{A} \kron \M{B}_{[n,m]} \\
    \end{bmatrix},
\end{equation}
%
where $\M{B}_{[i,j]}$ denotes a block of matrix $\M{B}$. The operator $\bkron$
is called block Kronecker product and is usually defined in a more general
form, where matrix $\M{A}$ is also partitioned \cite{Tracy1972sn,
Hyland1989jmaa, Koning1991laa}. For the purpose of the present work
partitioning of $\M{A}$ and nonuniform partitioning of $\M{B}$ are not
necessary and therefore not considered.


Block Kronecker product has several valuable properties, notably:
%
\begin{equation}
    (\M{I} \bkron \M{A})
    (\M{I} \bkron \M{B})
    =
    \M{I}
    \bkron
    (\M{A} \M{B})
    ,
\end{equation}
%
which allows for more efficient computations. Another commonly used property is
the following:
%
\begin{equation}
    \diag{k = 1...N}
    {
        \M{I} \kron \M{A}_k
    }
    =
    \M{I}
    \bkron
    \diag{k = 1...N}
    {
        \M{A}_k
    }
\end{equation}
%


For example, consider a time-invariant system with the state transition matrix
$\M{A} = \M{I} \kron \barM{A}$ and control matrix $\M{B} = \M{I} \kron
\barM{B}$. Then the condensing matrices $\M{U}_{x}$ and $\M{U}_{u}$ can be
expressed as
%
\begin{equation}
    \M{U}_{x} =
        \M{I}
        \bkron
        \barM{U}_{x}
        =
        \M{I}
        \bkron
        \begin{bmatrix}
            \barM{A}    \\
            \barM{A}^2  \\
            \vdots   \\
            \barM{A}^N  \\
        \end{bmatrix}
        =
        \begin{bmatrix}
            \M{I} \kron \barM{A}    \\
            \M{I} \kron \barM{A}^2  \\
            \vdots   \\
            \M{I} \kron \barM{A}^N  \\
        \end{bmatrix}
\end{equation}
%
\begin{equation}
    \M{U}_{u} =
        \M{I}
        \bkron
        \barM{U}_{u}
        =
        \M{I}
        \bkron
        \begin{bmatrix}
            \barM{B}                    & \barM{0}                  & \dots & \barM{0} \\
            \barM{A} \barM{B}           & \barM{B}                  & \dots & \barM{0} \\
            \vdots                      & \vdots                    & \ddots& \vdots \\
            \barM{A}^{N-1} \barM{B}     & \barM{A}^{N-2} \barM{B}   & \dots & \barM{B} \\
        \end{bmatrix}
\end{equation}
%
