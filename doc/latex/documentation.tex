\documentclass[12pt,a4paper]{book}

\usepackage{morewrites}

\usepackage{import}
\usepackage[sort]{standalone}

\usepackage{graphicx,color,psfrag}
\usepackage{tabularx,colortbl}

\usepackage{epsfig}


%\usepackage{listings}
%\usepackage{verbatim}   % useful for program listings
\usepackage{color}      % use if color is used in text
\usepackage{subfigure}  % use for side-by-side figures
\usepackage[colorlinks=true]{hyperref}   % use for hypertext links, including those to external documents and URLs

\usepackage{scalefnt}

%\usepackage{theorem}

\usepackage{tabu}
%\usepackage{forest}
\usepackage{cleveref}
\usepackage{xspace}
\usepackage[backend=bibtex,sorting=ydnt,firstinits=true,maxbibnames=99]{biblatex}

\usepackage[top=2cm, bottom=2cm, left=2cm, right=2cm]{geometry}

\usepackage{pgfplots}
\pgfplotsset{compat=newest}

\usepackage{placeins}


% number chapters in each part separately
\makeatletter
\@addtoreset{chapter}{part}
\makeatother


\addbibresource{bibliography.bib}
\addbibresource{wpg.bib}
\addbibresource{pepper.bib}

\usepackage{cancel}
\usepackage{longtable}
\usepackage[hyperref=true,list-style=longtable]{acro}
\usepackage{mymath}


% remove extra white space around long tables
\setlength{\LTpre}{4pt}
\setlength{\LTpost}{4pt}


\newcommand{\projectname}{\mono{Humoto}\xspace}
\newcommand{\sn}[1]{\mono{#1}}


\acsetup{list-short-format=\bf,list-heading=chapter*,list-name=List of Acronyms}
\acsetup{long-format=\emph}

\DeclareAcronym{API}{   short = API,    long = Application Programming Interface}
\DeclareAcronym{CPU}{   short = CPU,    long = Central Processing Unit}
\DeclareAcronym{CAS}{   short = CAS,    long = Computer Algebra System}
\DeclareAcronym{CoM}{   short = CoM,    long = Center of Mass}
\DeclareAcronym{CoP}{   short = CoP,    long = Center of Pressure}
\DeclareAcronym{FSM}{   short = FSM,    long = Finite State Machine}
\DeclareAcronym{KKT}{   short = KKT,    long = Karush-Kuhn-Tucker}
\DeclareAcronym{LIPM}{  short = LIPM,   long = Linear Inverted Pendulum Model}
\DeclareAcronym{MPC}{   short = MPC,    long = Model Predictive Control}
\DeclareAcronym{QP}{    short = QP,     long = Quadratic Program}
\DeclareAcronym{ZMP}{   short = ZMP,    long = Zero Moment Point}
\DeclareAcronym{PD}{    short = PD,     long = Proportional-Derivative}
\DeclareAcronym{WPG}{   short = WPG,    long = Walking Pattern Generator}
\DeclareAcronym{DS}{    short = DS ,    long = Double Support: both feet on the ground (see {\bf SS})}
\DeclareAcronym{TDS}{   short = TDS,    long = Transitional Double Support: {\bf DS} between adjacent {\bf LSS} and {\bf RSS}}
\DeclareAcronym{ADS}{   short = ADS,    long = Aligned Double Support: \bf{DS} in which the right and left feet are aligned}
\DeclareAcronym{SS}{    short = SS ,    long = Single Support: one foot on the ground (see {\bf DS})}
\DeclareAcronym{LSS}{   short = LSS,    long = Left Single Support (see {\bf SS})}
\DeclareAcronym{RSS}{   short = RSS,    long = Right Single Support (see {\bf SS})}


\sloppy


\newcommand{\x}{\V{\chi}}
\newcommand{\objb}{b}
\newcommand{\violation}{\V{v}}
\newcommand{\objA}{\M{\mathcal{A}}}


\setcounter{tocdepth}{3}
\setcounter{secnumdepth}{3}
\let\cleardoublepage\clearpage


\title{\projectname}
\author{}
\date{\today}


\begin{document}
\begin{refsection}
\maketitle
\tableofcontents

\clearpage
\phantomsection
%\pdfbookmark[1]{List of Acronyms}{listofacronyms}
\printacronyms
\addcontentsline{toc}{chapter}{List of Acronyms}


\part{Core}

\chapter{Introduction}


%%%%%%%%%%%%%%%%%%%%%%%%%%%%%%%%%%%%%%%%%%%%%%%%%%%%%%%%%%%%%%%%%%%%%%%%%%%%%%%%%%%%%%%%%%%%%%%%%%%%%%
%%%%%%%%%%%%%%%%%%%%%%%%%%%%%%%%%%%%%%%%%%%%%%%%%%%%%%%%%%%%%%%%%%%%%%%%%%%%%%%%%%%%%%%%%%%%%%%%%%%%%%
%%%%%%%%%%%%%%%%%%%%%%%%%%%%%%%%%%%%%%%%%%%%%%%%%%%%%%%%%%%%%%%%%%%%%%%%%%%%%%%%%%%%%%%%%%%%%%%%%%%%%%
\section{What is Humoto}\label{sec.intro}

\projectname{} -- is a software framework for manipulation of linear
    least-squares problems with (in)equality constraints. It supports both
    weighting and lexicographic prioritization and can be characterized as a
    tool for goal programming \cite{wiki2017gp}. However, the development was
    driven by works in other fields -- robotics, control, and numerical
    optimization, \EG, \cite{Dimitrov2015preprint, Saab2012tranrob,
    Escande2014ijrr}; for this reason our terminology and interpretations are
    different.


The core functionalities of \projectname are formulation of least-squares
    problems and their resolution using various third-party software. Both of
    these operations are performed through unified \ac{API}. Moreover, due to
    our interest in robotic applications, the framework facilitates formulation
    and implementation of optimization problems for control of robots, in
    particular, \ac{MPC} problems. For the same reason, we pay special
    attention to computational performance in order to be able to employ the
    framework in real-time scenarios.


In addition to the core components, the distribution of \projectname includes
    several modules -- implementations of specific controllers. The modules
    serve as examples of using the framework, but can also be used in
    accordance with their primary purpose. You can learn more about the
    provided modules in \cref{part.wpg} and \cref{part.pepper}.


%%%%%%%%%%%%%%%%%%%%%%%%%%%%%%%%%%%%%%%%%%%%%%%%%%%%%%%%%%%%%%%%%%%%%%%%%%%%%%%%%%%%%%%%%%%%%%%%%%%%%%
%%%%%%%%%%%%%%%%%%%%%%%%%%%%%%%%%%%%%%%%%%%%%%%%%%%%%%%%%%%%%%%%%%%%%%%%%%%%%%%%%%%%%%%%%%%%%%%%%%%%%%
%%%%%%%%%%%%%%%%%%%%%%%%%%%%%%%%%%%%%%%%%%%%%%%%%%%%%%%%%%%%%%%%%%%%%%%%%%%%%%%%%%%%%%%%%%%%%%%%%%%%%%
\section{Outline}\label{sec.intro}

The present document introduces basic concepts behind the \projectname and
    contains various mathematical derivations which are used in the framework
    and the modules. However, this document cannot serve as a comprehensive
    guide -- you have to refer to \sn{Doxygen} documentation in order to learn
    how to use \projectname in your applications or learn more about
    implementation details.



%%%%%%%%%%%%%%%%%%%%%%%%%%%%%%%%%%%%%%%%%%%%%%%%%%%%%%%%%%%%%%%%%%%%%%%%%%%%%%%%%%%%%%%%%%%%%%%%%%%%%%
%%%%%%%%%%%%%%%%%%%%%%%%%%%%%%%%%%%%%%%%%%%%%%%%%%%%%%%%%%%%%%%%%%%%%%%%%%%%%%%%%%%%%%%%%%%%%%%%%%%%%%
%%%%%%%%%%%%%%%%%%%%%%%%%%%%%%%%%%%%%%%%%%%%%%%%%%%%%%%%%%%%%%%%%%%%%%%%%%%%%%%%%%%%%%%%%%%%%%%%%%%%%%
\section{Notation}\label{sec.notation}

\begin{description}
    \item[Software names] \hfill \\
        Names of programs and software libraries, names of constants, variables
        and functions that are used in programs are typed in a monospaced font:
        \sn{Eigen}.

    \item[Scalars, vectors, matrices] \hfill
        \begin{itemize}
            \item Vectors and matrices are denoted by letters in a bold font:
                $\V{v}$, $\M{M}$, $\M{\mathcal{A}}$.

            \item Scalars are denoted using the standard italic or calligraphic
                font: $N, n, \mathcal{K}$.

            \item $\T{(\cdot)}$ -- transpose of a matrix or a vector.

            \item $\CROSS[(\cdot)]$ -- a skew-symmetric matrix used for
                representation of a cross product of two three dimensional
                vectors as a product of a matrix and a vector:
                %
                \begin{equation}
                    \V{v}
                    =
                    \begin{bmatrix}
                        v^x\\
                        v^y\\
                        v^z
                    \end{bmatrix},
                    \quad
                    \CROSS[\V{v}]
                    =
                    \begin{bmatrix}
                        0      &   -v^z   &   v^y\\
                        v^z    &   0      &   -v^x\\
                        -v^y   &   v^x    &   0 \\
                    \end{bmatrix}.
                \end{equation}

            \item Block diagonal matrices:
                %
                \begin{equation}
                    \begin{gathered}
                        \diag{2}{\M{M}} =
                        \begin{bmatrix}
                            \M{M}   &   \M{0}\\
                            \M{0}   &   \M{M}\\
                        \end{bmatrix}
                        ,
                        \quad
                        \diag{k = 1 \dots 2}{\M{M}_k} =
                        \begin{bmatrix}
                            \M{M}_1 &   \M{0}  \\
                            \M{0}   &   \M{M}_2\\
                        \end{bmatrix}
                        ,
                        \\
                        \diag{}{\M{M}, \M{R}} =
                        \begin{bmatrix}
                            \M{M} &   \M{0}  \\
                            \M{0}   &   \M{R}\\
                        \end{bmatrix}
                        .
                    \end{gathered}
               \end{equation}

            \item Stacked vectors and matrices:
                %
                \begin{equation}
                    \V{v} = (\V{v}_1, \dots, \V{v}_n) = \begin{bmatrix} \V{v}_1 \\ \vdots \\ \V{v}_n \\ \end{bmatrix},
                    \quad
                    \M{M} = (\M{M}_1, \dots, \M{M}_n) = \begin{bmatrix} \M{M}_1 \\ \vdots \\ \M{M}_n \\ \end{bmatrix}.
                \end{equation}

            \item Inequalities between vectors $\V{v} \ge \V{r}$ are interpreted
                component-wise.
        \end{itemize}


    \item[Special matrices and vectors] \hfill
        \begin{itemize}
            \item $\M{I}$ -- an identity matrix. $\M{I}_n$ -- $n \CROSS n$ identity matrix.

            \item $\M{I}_{(\cdot)}$ -- a selection matrix.

            \item $\M{0}$ -- a matrix of zeros. $\M{0}_{n,m}$ -- $n \CROSS m$ matrix of zeros.
        \end{itemize}


    \item[Reference frames] \hfill
        \begin{itemize}
            \item Frames are denoted using a sans-serif font: $\FRAME{A}$. All
                considered frames are orthonormal.

            \item $\V[A]{v}$ -- vector expressed in frame $\FRAME{A}$.

            \item $\M[B][A]{R}$ -- rotation matrix from from frame $\FRAME{B}$
                to frame $\FRAME{A}$. For example, 2-d rotation matrix is defined
                as
                %
                \begin{equation}
                    \M[B][A]{R}
                    =
                    \begin{bmatrix}
                        \cos(\theta) &   -\sin(\theta)\\
                        \sin(\theta) &   \cos(\theta)\\
                    \end{bmatrix},
                \end{equation}
                %
                where $\theta$ is the rotation angle of frame $\FRAME{B}$ with
                respect to $\FRAME{A}$.

            \item The global frame is implicit and is not denoted by any
                letter, \EG, $\M[B][]{R}$ rotates from frame $\FRAME{B}$ to the
                global frame.
        \end{itemize}


    \item[Sets] \hfill
        \begin{itemize}
            \item The sets are denoted using a blackboard bold font: $\SET{A}$.

            \item $\RR$ is the set of real numbers.

            \item $\RR_{\ge 0}, \RR_{> 0}$ are the sets of non-negative and positive real numbers.

            \item $\RR^n$ is the set of real-valued vectors.

            \item $\RR^{n \CROSS m}$ is the set of real-valued matrices.
        \end{itemize}


    \item[Other] \hfill
        \begin{itemize}
            \item Function names in mathematical expressions are written in the
                regular font: $\FUNC{func}(\V{x}, \V{y})$.

            \item $\NORME{\cdot}$ denotes the Euclidean norm.
        \end{itemize}
\end{description}


%%%%%%%%%%%%%%%%%%%%%%%%%%%%%%%%%%%%%%%%%%%%%%%%%
%%%%%%%%%%%%%%%%%%%%%%%%%%%%%%%%%%%%%%%%%%%%%%%%%
%%%%%%%%%%%%%%%%%%%%%%%%%%%%%%%%%%%%%%%%%%%%%%%%%

\chapter{Optimization concepts}

This chapter introduces basic concepts and terms which commonly used in the
framework. The presentation is terse, so you may be interested in reading more
in other sources \cite{Sherikov2016phd, Dimitrov2015preprint, Saab2012tranrob,
Escande2014ijrr}.



%%%%%%%%%%%%%%%%%%%%%%%%%%%%%%%%%%%%%%%%%%%%%%%%%%%%%%%%%%%%%%%%%%%%%%%%%%%%%%%%%%%%%%%%%%%%%%%%%%%%%%
%%%%%%%%%%%%%%%%%%%%%%%%%%%%%%%%%%%%%%%%%%%%%%%%%%%%%%%%%%%%%%%%%%%%%%%%%%%%%%%%%%%%%%%%%%%%%%%%%%%%%%
%%%%%%%%%%%%%%%%%%%%%%%%%%%%%%%%%%%%%%%%%%%%%%%%%%%%%%%%%%%%%%%%%%%%%%%%%%%%%%%%%%%%%%%%%%%%%%%%%%%%%%
\section{Tasks}\label{sec.tasks}

We define a \emph{task} as set of logically related constraints of the form
%
\begin{equation}
    \ubarV{\objb}
    \le
    \objA \x
    \le
    \barV{\objb},
\end{equation}
%
where $\objA \in \RR^{n \times m}$ is a given matrix; $\ubarV{\objb} \in \RR^n$
and $\barV{\objb} \in \RR^n$ are given vectors of lower and upper bounds, which
may include infinitely large values; and $\x \in \RR^m$ is a vector of decision
variables. We consider only well defined tasks where $\ubarV{\objb} \le
\barV{\objb}$ is always true.


Usually, control or optimization problems are composed of multiple tasks, which
may not be achieved exactly. In order to account for this we assume that there
exist implicit vector of violations $\violation \in \RR^n$ such that expression
%
\begin{equation}
    \ubarV{\objb}
    \le
    \objA \x
    -
    \violation
    \le
    \barV{\objb}
\end{equation}
%
is always exactly satisfied. A value in vector $\violation$ is negative when
the corresponding lower bound is violated, positive when the upper bound is
violated, and zero when the constraint is satisfied exactly. Here we aim at
satisfaction of a task in the least-squares sense, which is equivalent to
solving the following \ac{QP}
%
\begin{equation}
    \begin{aligned}
        \MINIMIZE{\x, \violation}   & \NORME{\violation}^2 \\
        \SUBJECTTO                  & \ubarV{\objb} \le \objA \x  -  \violation \le \barV{\objb}
    \end{aligned}
\end{equation}
%
for optimal $\violation^{\star}$. Note that due to semidefinite nature of this
problem $\x^{\star}$ is not necessarily unique.



%%%%%%%%%%%%%%%%%%%%%%%%%%%%%%%%%%%%%%%%%%%%%%%%%%%%%%%%%%%%%%%%%%%%%%%%%%%%%%%%%%%%%%%%%%%%%%%%%%%%%%
%%%%%%%%%%%%%%%%%%%%%%%%%%%%%%%%%%%%%%%%%%%%%%%%%%%%%%%%%%%%%%%%%%%%%%%%%%%%%%%%%%%%%%%%%%%%%%%%%%%%%%
%%%%%%%%%%%%%%%%%%%%%%%%%%%%%%%%%%%%%%%%%%%%%%%%%%%%%%%%%%%%%%%%%%%%%%%%%%%%%%%%%%%%%%%%%%%%%%%%%%%%%%
\section{Objectives: weighting of tasks}

Consider two tasks
%
\begin{equation}
    \ubarV{\objb}_1 \le \objA_1 \x \le \barV{\objb}_1
    \quad
    \mbox{and}
    \quad
    \ubarV{\objb}_2 \le \objA_2 \x \le \barV{\objb}_2,
\end{equation}
%
which are known to be in a conflict, \IE, cannot be satisfied exactly with zero
$\violation_1$ and $\violation_2$ simultaneously. In this case, we may still
want to satisfy the tasks simultaneously as much as possible with a certain
trade-off, which can be achieved with minimizaion of a weighted sum of the
norms of violations:
%
\begin{equation}
    \gamma_1
    \NORME{\violation_1}
    +
    \gamma_2
    \NORME{\violation_2},
\end{equation}
%
where $\gamma_1 \in \RR_{\ge 0}$ and $\gamma_2 \in \RR_{\ge 0}$. For example,
$\gamma_1 > \gamma_2$ gives higher priority to the first task.


We call a group of weighted tasks an \emph{objective}; note, however, that the
difference between a ``task'' and ``objective'' is purely terminological.
Usually, the weights are ommited since the necessary effect can be achieved by
scaling the task.



%%%%%%%%%%%%%%%%%%%%%%%%%%%%%%%%%%%%%%%%%%%%%%%%%%%%%%%%%%%%%%%%%%%%%%%%%%%%%%%%%%%%%%%%%%%%%%%%%%%%%%
%%%%%%%%%%%%%%%%%%%%%%%%%%%%%%%%%%%%%%%%%%%%%%%%%%%%%%%%%%%%%%%%%%%%%%%%%%%%%%%%%%%%%%%%%%%%%%%%%%%%%%
%%%%%%%%%%%%%%%%%%%%%%%%%%%%%%%%%%%%%%%%%%%%%%%%%%%%%%%%%%%%%%%%%%%%%%%%%%%%%%%%%%%%%%%%%%%%%%%%%%%%%%
\section{Hierarchies: prioritization of tasks (objectives)}

In the previous section we considered case when we want to satisfy conflicting
tasks simultaneously, but what if one of the tasks has infinitely higher
priority than another? In this case we resort to \emph{hierarchies} of tasks or
objectives:
%
\begin{hierarchy}
    \level $\ubarV{\objb}_1 \le \objA_1 \x \le \barV{\objb}_1$
    \level $\ubarV{\objb}_2 \le \objA_2 \x \le \barV{\objb}_2$
    \level $\ubarV{\objb}_3 \le \objA_3 \x \le \barV{\objb}_3$
    \levelLabel{$\dots$} $\dots$
\end{hierarchy}
%
Here task (objective) $i$ is infinitely more important than task $i+1$, \IE,
satisfaction of task $i+1$ must never come at a price of increasing
$\NORME{\violation_i}^2$.


A hierarchy can be solved using a sequence of \ac{QP} or with the help of a
specialized algorithm~\cite{Dimitrov2015preprint}.



%%%%%%%%%%%%%%%%%%%%%%%%%%%%%%%%%%%%%%%%%%%%%%%%%%%%%%%%%%%%%%%%%%%%%%%%%%%%%%%%%%%%%%%%%%%%%%%%%%%%%%
%%%%%%%%%%%%%%%%%%%%%%%%%%%%%%%%%%%%%%%%%%%%%%%%%%%%%%%%%%%%%%%%%%%%%%%%%%%%%%%%%%%%%%%%%%%%%%%%%%%%%%
%%%%%%%%%%%%%%%%%%%%%%%%%%%%%%%%%%%%%%%%%%%%%%%%%%%%%%%%%%%%%%%%%%%%%%%%%%%%%%%%%%%%%%%%%%%%%%%%%%%%%%
\section{Relation between hierarchies and QPs}

Note that a \ac{QP} can be represented as a hierarchy with two levels
%
\begin{hierarchy}
    \level $\ubarV{\objb}_1 \le \objA_1 \x \le \barV{\objb}_1$
    \level $\objA_2 \x = \V{\objb}_2$
\end{hierarchy}
%
where objective on the second level is an equality. The only difference is that
\ac{QP} requires the first inequality objective to be feasible. Since this
condition is satisfied in many applications, the framework allows to cast and
solve a hierarchy of two levels as a single \ac{QP}, even though this is not
strictly correct. This behavior can be suppressed by changing parameters of the
solvers.



%%%%%%%%%%%%%%%%%%%%%%%%%%%%%%%%%%%%%%%%%%%%%%%%%%%%%%%%%%%%%%%%%%%%%%%%%%%%%%%%%%%%%%%%%%%%%%%%%%%%%%
%%%%%%%%%%%%%%%%%%%%%%%%%%%%%%%%%%%%%%%%%%%%%%%%%%%%%%%%%%%%%%%%%%%%%%%%%%%%%%%%%%%%%%%%%%%%%%%%%%%%%%
%%%%%%%%%%%%%%%%%%%%%%%%%%%%%%%%%%%%%%%%%%%%%%%%%%%%%%%%%%%%%%%%%%%%%%%%%%%%%%%%%%%%%%%%%%%%%%%%%%%%%%
\section{Computational performance}


%%%%%%%%%%%%%%%%%%%%%%%%%%%%%%%%%%%%%%%%%%%%%%%%%%%%%%%%%%%%%%%%%%%%%%%%%%%%%%%%%%%%%%%%%%%%%%%%%%%%%%
\subsection{Hot-starting}

Many solvers for optimization problems support hot-starting -- they accept
additional data, which may help to reduce computation time. Currently the
framework allows hot-starting using
%
\begin{itemize}
    \item a guess of the set of constraints, which are active at the solution;
    \item a guess of the solution.
\end{itemize}
%


%%%%%%%%%%%%%%%%%%%%%%%%%%%%%%%%%%%%%%%%%%%%%%%%%%%%%%%%%%%%%%%%%%%%%%%%%%%%%%%%%%%%%%%%%%%%%%%%%%%%%%
\subsection{Exploitation of the problem structure}

One of the ways to improve performance of the solver is to shape the
optimization problem in a beneficial manner and to inform the solver about the
structure of the problem.


%%%%%%%%%%%%%%%%%%%%%%%%%%%%%%%%%%%%%%%%%%
\subsubsection{Two-sided inequalities}

If a task is bounded from both sides it is beneficial to express it in the
following form
%
\begin{equation}
    \ubarV{\objb}
    \le
    \objA \x
    \le
    \barV{\objb}
\end{equation}
%
instead of splitting it into two parts corresponding to lower and upper bounds
as is common in practice:
%
\begin{equation}
    \begin{bmatrix}
        \objA\\
        -\objA\\
    \end{bmatrix}
    \x
    \le
    \begin{bmatrix}
        \barV{\objb}\\
        -\ubarV{\objb}
    \end{bmatrix}
\end{equation}
%
The reason for this is that bounds $\ubarV{\objb} < \barV{\objb}$ cannot be
violated simultaneously, which can be exploited by a solver to reduce
computational load.


%%%%%%%%%%%%%%%%%%%%%%%%%%%%%%%%%%%%%%%%%%
\subsubsection{Sparsity}

We call a task \emph{sparse} if the corresponding matrices and vectors contain
a large number of zeros. A task with simple bounds (box constraints) on the
decision variables
%
\begin{equation}
    \ubarV{\objb}
    \le
    \x
    \le
    \barV{\objb}
\end{equation}
%
is a typical example of a sparse task. Handling of such constraints can be
implemented in a very efficient way and is supported by many solvers. It is
often beneficial to reformulate an optimization problem in order to express
inequality tasks with simple bounds.

The framework does not support generic sparse matrices, but only specific
sparsity types listed in the table below
%
{\tabulinesep=1.2mm
\begin{longtabu}{c | c | c | c | c}
    Equality (zero)         &   Equality                &   Lower bounds                    &   Upper bounds                    &   Lower and upper bounds \\
    \hline
    $\objA \x = \V{0}$      &   $\objA \x = \V{\objb}$  &
                                                          $\ubarV{\objb} \le \objA \x$      &
                                                                                            $\objA \x \le \barV{\objb}$         &   $\ubarV{\objb} \le \objA \x \le \barV{\objb}$ \\
    \hline
    $\objA\M{S} \x = \V{0}$ &   $\objA\M{S} \x = \V{\objb}$ &
                                                            $\ubarV{\objb} \le \objA\M{S} \x$   &
                                                                                                $\objA\M{S} \x \le \barV{\objb}$    &   $\ubarV{\objb} \le \objA\M{S} \x \le \barV{\objb}$ \\
    \hline
    $\M{G} \x = \V{0}$      &   $\M{G} \x = \V{\objb}$  &   $\ubarV{\objb} \le \M{G} \x$    &   $\M{G} \x \le \barV{\objb}$     &   $\ubarV{\objb} \le \M{G} \x \le \barV{\objb}$ \\
    \hline
    $\M{I} \x = \V{0}$      &   $\M{I} \x = \V{\objb}$  &   $\ubarV{\objb} \le \M{I} \x$    &   $\M{I} \x \le \barV{\objb}$     &   $\ubarV{\objb} \le \M{I} \x \le \barV{\objb}$ \\
\end{longtabu}
}
%
\noindent Here $\M{S}$ selects a continuous segment of $\x$; $\M{G}$ is a
weighted selection matrix; $\M{I}$ is a simple selection matrix. Note that not
all sparsity types are supported by the solvers supported by the framework.



%%%%%%%%%%%%%%%%%%%%%%%%%%%%%%%%%%%%%%%%%%%%%%%%%%%%%%%%%%%%%%%%%%%%%%%%%%%%%%%%
\subsection{Early termination}

Some solvers support early termination by imposing a limit on the number of
iterations or computation time. Early termination is potentially dangerous
since the solution returned by the solver is suboptimal, \IE, some feasible
tasks may not be satisfied.




%%%%%%%%%%%%%%%%%%%%%%%%%%%%%%%%%%%%%%%%%%%%%%%%%
%%%%%%%%%%%%%%%%%%%%%%%%%%%%%%%%%%%%%%%%%%%%%%%%%
%%%%%%%%%%%%%%%%%%%%%%%%%%%%%%%%%%%%%%%%%%%%%%%%%

\chapter{Reference}

This chapter contains various derivations and definitions which are used by
multiple modules of \projectname.


%%%%%%%%%%%%%%%%%%%%%%%%%%%%%%%%%%%%%%%%%%%%%%%%%%%%%%%%%%%%%%%%%%%%%%%%%%%%%%%%%%%%%%%%%%%%%%%%%%%%%%
%%%%%%%%%%%%%%%%%%%%%%%%%%%%%%%%%%%%%%%%%%%%%%%%%%%%%%%%%%%%%%%%%%%%%%%%%%%%%%%%%%%%%%%%%%%%%%%%%%%%%%
%%%%%%%%%%%%%%%%%%%%%%%%%%%%%%%%%%%%%%%%%%%%%%%%%%%%%%%%%%%%%%%%%%%%%%%%%%%%%%%%%%%%%%%%%%%%%%%%%%%%%%
\section{Model Predictive Control}\label{sec.condensing}

\acf{MPC} is a branch of control theory, where the control inputs are generated
by optimizing behavior of a system over certain preview horizon.

A linear model of the system has the following form:
%
\begin{equation}
        \V{x}_{k+1} = \M{A}_k \V{x}_k + \M{B}_k \V{u}_k, \quad k = 0, \dots, N
\end{equation}
%
where $\V{x}_k$ and $\V{u}_k$ are $k$-th state and control input respectively,
while $N$ is the length of preview (prediction) horizon.

Ouptput of the system can be defined in different ways, here we assume that it
depends on the preceding state and control
%
\begin{equation}
        \V{y}_{k+1} = \M{D}_k \V{x}_k + \M{E}_k \V{u}_k, \quad k = 0, \dots, N
\end{equation}
%


%%%%%%%%%%%%%%%%%%%%%%%%%%%%%%%%%%%%%%%%%%%%%%%%%%%%%%%%%%%%%%%%%%%%%%%%%%%%%%%%%%%%%%%%%%%%%%%%%%%%%%
\subsection{Condensing}\label{sec.condensing}

Condensing amounts to finding such matrices $\M{U}_{x}$ and $\M{U}_{u}$ that
%
\begin{equation}
    \V{v}_{x} = \M{U}_{x} \V{x}_0 + \M{U}_{u} \V{v}_{u},
\end{equation}
%
where
%
\begin{equation}
\begin{split}
    \V{v}_{x} &= (\V{x}_1, \dots, \V{x}_N), \\
    \V{v}_{u} &= (\V{u}_0, \dots, \V{u}_{N-1}).
\end{split}
\end{equation}
%

Similarly for the output
\begin{equation}
    \V{v}_{y} = \M{O}_{x} \V{x}_0 + \M{O}_{u} \V{v}_{u}.
\end{equation}
Note that
\begin{equation}
    \M{O}_{x}
    =
    \begin{bmatrix}
        \diag{k=0 \dots N-1}{D_k}  &  \M{0} \\
    \end{bmatrix}
    \begin{bmatrix}
        \M{0} \\
        \M{U}_{x} \\
    \end{bmatrix}
    \quad
    \quad
    \M{O}_{u}
    =
    \begin{bmatrix}
        \diag{k=0 \dots N-1}{D_k}  &  \M{0} \\
    \end{bmatrix}
    \begin{bmatrix}
        \M{0} \\
        \M{U}_{u} \\
    \end{bmatrix}
    +
    \diag{k=0 \dots N-1}{E_k}
    \M{U}_{u}
\end{equation}


%%%%%%%%%%%%%%%%%%%%%%%%%%%%%%%%%%%%%%%%%%%%%%%%%%%%%%%%%%%%%%%%%%%%%%%%%%%%%%%%%%%%%%%%%%%%%%%%%%%%%%
\subsubsection{Time-variant system}\label{sec.condensing_gen}
\begin{equation}
    \M{U}_{x} =
        \begin{bmatrix}
        \M{A}_0    \\
        \M{A}_1 \M{A}_0  \\
        \vdots           \\
        \M{A}_{N-1} \dots \M{A}_0 \\
        \end{bmatrix}
    \quad\quad
    \M{U}_{u} =
        \begin{bmatrix}
        \M{B}_0                             & \M{0}                                 & \dots & \M{0} \\
        \M{A}_1 \M{B}_0                     & \M{B}_1                               & \dots & \M{0} \\
        \vdots                              & \vdots                                & \ddots& \vdots \\
        \M{A}_{N-1} \dots \M{A}_1 \M{B}_0   & \M{A}_{N-1} \dots \M{A}_2 \M{B}_1     & \dots & \M{B}_{N-1} \\
        \end{bmatrix}
\end{equation}

\begin{equation}
    \M{O}_{x} =
        \begin{bmatrix}
        \M{D}_0    \\
        \M{D}_1 \M{A}_0    \\
        \M{D}_2 \M{A}_1 \M{A}_0  \\
        \vdots           \\
        \M{D}_{N-1} \M{A}_{N-2} \dots \M{A}_0 \\
        \end{bmatrix}
    \quad\quad
    \M{O}_{u} =
        \begin{bmatrix}
        \M{E}_0                             & \M{0}                                 & \dots & \M{0} \\
        \M{D}_1 \M{B}_0                     & \M{E}_1                               & \dots & \M{0} \\
        \M{D}_2 \M{A}_1 \M{B}_0             & \M{D}_2 \M{B}_1                       & \dots & \M{0} \\
        \vdots                              & \vdots                                & \ddots& \vdots \\
        \M{D}_{N-1} \M{A}_{N-2} \dots \M{A}_1 \M{B}_0   & \M{D}_{N-1} \M{A}_{N-2} \dots \M{A}_2 \M{B}_1     & \dots & \M{E}_{N-1} \\
        \end{bmatrix}
\end{equation}


%%%%%%%%%%%%%%%%%%%%%%%%%%%%%%%%%%%%%%%%%%%%%%%%%%%%%%%%%%%%%%%%%%%%%%%%%%%%%%%%%%%%%%%%%%%%%%%%%%%%%%
\subsubsection{System is varying in the first preview interval}
\begin{equation}
    \M{U}_{x} =
        \begin{bmatrix}
        \M{A}_0    \\
        \M{A} \M{A}_0  \\
        \vdots   \\
        \M{A}^{N-1} \M{A}_0 \\
        \end{bmatrix}
    \quad\quad
    \M{U}_{u} =
        \begin{bmatrix}
        \M{B}_0                 & \M{0}                 & \dots & \M{0} \\
        \M{A} \M{B}_0           & \M{B}                 & \dots & \M{0} \\
        \vdots                  & \vdots                & \ddots& \vdots \\
        \M{A}^{N-1} \M{B}_0     & \M{A}^{N-2} \M{B}     & \dots & \M{B} \\
        \end{bmatrix}
\end{equation}

\begin{equation}
    \M{O}_{x} =
        \begin{bmatrix}
        \M{D}_0    \\
        \M{D} \M{A}_0    \\
        \M{D} \M{A} \M{A}_0  \\
        \vdots           \\
        \M{D} \M{A}^{N-2} \M{A}_0 \\
        \end{bmatrix}
    \quad\quad
    \M{O}_{u} =
        \begin{bmatrix}
        \M{E}_0                             & \M{0}                                 & \dots & \M{0} \\
        \M{D} \M{B}_0                       & \M{E}                                 & \dots & \M{0} \\
        \M{D} \M{A} \M{B}_0                 & \M{D} \M{B}                           & \dots & \M{0} \\
        \vdots                              & \vdots                                & \ddots& \vdots \\
        \M{D} \M{A}^{N-2} \M{B}_0           & \M{D} \M{A}^{N-3} \M{B}               & \dots & \M{E} \\
        \end{bmatrix}
\end{equation}

%%%%%%%%%%%%%%%%%%%%%%%%%%%%%%%%%%%%%%%%%%%%%%%%%%%%%%%%%%%%%%%%%%%%%%%%%%%%%%%%%%%%%%%%%%%%%%%%%%%%%%
\subsubsection{Time-invariant system}
\begin{equation}
    \M{U}_{x} =
        \begin{bmatrix}
        \M{A}    \\
        \M{A}^2  \\
        \vdots   \\
        \M{A}^N  \\
        \end{bmatrix}
    \quad\quad
    \M{U}_{u} =
        \begin{bmatrix}
        \M{B}                   & \M{0}                 & \dots & \M{0} \\
        \M{A} \M{B}             & \M{B}                 & \dots & \M{0} \\
        \vdots                  & \vdots                & \ddots& \vdots \\
        \M{A}^{N-1} \M{B}       & \M{A}^{N-2} \M{B}     & \dots & \M{B} \\
        \end{bmatrix}
\end{equation}


\begin{equation}
    \M{O}_{x} =
        \begin{bmatrix}
        \M{D}    \\
        \M{D} \M{A}    \\
        \M{D} \M{A}^2  \\
        \vdots           \\
        \M{D} \M{A}^{N-1} \\
        \end{bmatrix}
    \quad\quad
    \M{O}_{u} =
        \begin{bmatrix}
        \M{E}                               & \M{0}                                 & \dots & \M{0} \\
        \M{D} \M{B}                         & \M{E}                                 & \dots & \M{0} \\
        \M{D} \M{A} \M{B}                   & \M{D} \M{B}                           & \dots & \M{0} \\
        \vdots                              & \vdots                                & \ddots& \vdots \\
        \M{D} \M{A}^{N-2} \M{B}             & \M{D} \M{A}^{N-3} \M{B}               & \dots & \M{E} \\
        \end{bmatrix}
\end{equation}



%%%%%%%%%%%%%%%%%%%%%%%%%%%%%%%%%%%%%%%%%%%%%%%%%%%%%%%%%%%%%%%%%%%%%%%%%%%%%%%%
%%%%%%%%%%%%%%%%%%%%%%%%%%%%%%%%%%%%%%%%%%%%%%%%%%%%%%%%%%%%%%%%%%%%%%%%%%%%%%%%
%%%%%%%%%%%%%%%%%%%%%%%%%%%%%%%%%%%%%%%%%%%%%%%%%%%%%%%%%%%%%%%%%%%%%%%%%%%%%%%%
\section{Triple integrator (discrete-time)}

In some cases it is convenient to represent model of the system with triple
integrator
%
\begin{equation}
    \V{x}_{k+1}
    =
    \M{A}_k
    \V{x}_{k}
    +
    \M{B}_k
    \dddot{x}_{k},
    \quad
    \V{x}_{k}
    =
    (
        x_k,
        \dot{x}_k,
        \ddot{x}_k
    )
\end{equation}
%
%
\begin{equation}
    \M{A}_k
    =
    \begin{bmatrix}
        1 & T_k & \frac{T_k^2}{2}\\
        0 & 1 & T_k\\
        0 & 0 & 1
    \end{bmatrix}
    ,
    \quad
    \M{B}_k
    =
    \begin{bmatrix}
        \frac{T_k^3}{6} \\
        \frac{T_k^2}{2} \\
        T_k
    \end{bmatrix}
\end{equation}
%


The corresponding matrices are defined in \sn{Maxima} as
%
\begin{listingtcb}{Maxima}
\begin{deflisting}
A: matrix([1, T_k, T_k^2/2], [0, 1, T_k], [0, 0, 1]);
B: matrix([T_k^3/6], [T_k^2/2], [T_k]);
X0: matrix([x0],[dx0],[ddx0]);
U: matrix([dddx0]);
X1: matrix([x1],[dx1],[ddx1]);
As: matrix([1, T_s, T_s^2/2], [0, 1, T_s], [0, 0, 1]);
Bs: matrix([T_s^3/6], [T_s^2/2], [T_s]);
\end{deflisting}
\end{listingtcb}
%


It is possible to reformulate the triple integrator model in order to use
acceleration, velocity, or position from the next state as control input
instead of the jerk. This may be useful in the cases when constraints on the
state variables can be represented with simple bounds. The adjusted model
produces the same motions, but numerical properties of its matrices may be
different.


%%%%%%%%%%%%%%%%%%%%%%%%%%%%%%%%%%%%%%%%%%%%%%%%%%%%%%%%%%%%%%%%%%%%%%%%%%%%%%%%
\subsection{Controlled using acceleration}

\begin{align}
    \V{x}_{k+1}
    &=
    \M{A}_k
    \V{x}_{k}
    +
    \M{B}_k
    \ddot{x}_{k+1}
    \\
    \dddot{x}_{k}
    &=
    \M{D}_k
    \V{x}_{k}
    +
    \M{E}_k
    \ddot{x}_{k+1}
\end{align}


%
\begin{equation}
    \M{A}_k =
        \begin{bmatrix}
            1   & T_k   & \frac{T_k^2}{3} \\
            0   & 1     & \frac{T_k}{2} \\
            0   & 0     & 0 \\
        \end{bmatrix},
    \quad
    \M{B}_k =
        \begin{bmatrix}
            \frac{T_k^2}{6} \\
            \frac{T_k}{2} \\
            1 \\
        \end{bmatrix},
    \quad
    \M{D}_k =
        \begin{bmatrix}
            0 & 0 & - \frac{1}{T_k}
        \end{bmatrix},
    \quad
    \M{E}_k =
        \begin{bmatrix}
            \frac{1}{T_k} \\
        \end{bmatrix}
\end{equation}
%

For $T_s \in [0, T_k]$
%
\begin{equation}
    \M{A}_{s,k} =
        \begin{bmatrix}
            1 & {T_s}   &  - \frac{{T_s}^3 - 3 {T_k} {T_s}^2}{6 {T_k}} \\
            0 & 1       &  - \frac{{T_s}^2 - 2 {T_k} {T_s}}{2 {T_k}} \\
            0 & 0       &  - \frac{{T_s} - {T_k}}{{T_k}} \\
        \end{bmatrix}
    ,
    \quad
    \M{B}_{s,k} =
        \begin{bmatrix}
            \frac{{T_s}^3}{6 {T_k}} \\
            \frac{{T_s}^2}{2 {T_k}} \\
            \frac{{T_s}}{{T_k}} \\
        \end{bmatrix}
\end{equation}
%


\begin{listingtcb}{Maxima}
\begin{deflisting}
e: solve([(A.X0 + B.U)[3][1] = (X1)[3][1]], [dddx0]);
Dnew: coefmatrix([rhs(e[1])], list_matrix_entries(X0));
Enew: coefmatrix([rhs(e[1])], [ddx1]);
e: subst(e, A.X0 + B.U);
Anew: coefmatrix(list_matrix_entries(e), list_matrix_entries(X0));
Bnew: coefmatrix(list_matrix_entries(e), [ddx1]);
tex(Anew);
tex(Bnew);
tex(Dnew);
tex(Enew);
/* subsampling */
e: solve([(A.X0 + B.U)[3][1] = (X1)[3][1]], [dddx0]);
e: subst(e, As.X0 + Bs.U);
Asnew: coefmatrix(list_matrix_entries(e), list_matrix_entries(X0));
Bsnew: coefmatrix(list_matrix_entries(e), [ddx1]);
tex(Asnew);
tex(Bsnew);
/* check */
subst([T_s = T_k], Asnew) - Anew;
subst([T_s = T_k], Bsnew) - Bnew;
\end{deflisting}
\end{listingtcb}



%%%%%%%%%%%%%%%%%%%%%%%%%%%%%%%%%%%%%%%%%%%%%%%%%%%%%%%%%%%%%%%%%%%%%%%%%%%%%%%%
\subsection{Controlled using velocity}

%
\begin{align}
    \V{x}_{k+1}
    &=
    \M{A}_k
    \V{x}_{k}
    +
    \M{B}_k
    \dot{x}_{k+1}
    \\
    \dddot{x}_{k}
    &=
    \M{D}_k
    \V{x}_{k}
    +
    \M{E}_k
    \dot{x}_{k+1}
\end{align}
%

%
\begin{equation}
    \M{A}_k =
        \begin{bmatrix}
            1     & \frac{2 T_k}{3}     & \frac{T_k^2}{6} \\
            0     & 0                   & 0 \\
            0     & -\frac{2}{T_k}      &  - 1 \\
        \end{bmatrix},
    \quad
    \M{B}_k =
        \begin{bmatrix}
            \frac{T_k}{3} \\
            1 \\
            \frac{2}{T_k} \\
        \end{bmatrix},
    \quad
    \M{D}_k =
        \begin{bmatrix}
            0 & -\frac{2}{T_k^2} & -\frac{2}{T_k} \\
        \end{bmatrix},
    \quad
    \M{E}_k =
        \begin{bmatrix}
            \frac{2}{T_k^2} \\
        \end{bmatrix}
\end{equation}
%


For $T_s \in [0, T_k]$
%
\begin{equation}
    \M{A}_{s,k} =
        \begin{bmatrix}
            1 &  - \frac{{T_s}^3 - 3  {T_k}^2 {T_s}}{3 {T_k}^2} &  - \frac{2 {T_s}^3 - 3 {T_k} {T_s}^2}{6 {T_k}} \\
            0 &  - \frac{{T_s}^2 - {T_k}^2}{{T_k}^2}            &  - \frac{{T_s}^2 - {T_k} {T_s}}{{T_k}} \\
            0 &  - \frac{2 {T_s}}{{T_k}^2}                      &  - \frac{2 {T_s} - {T_k}}{{T_k}} \\
        \end{bmatrix}
    ,
    \quad
    \M{B}_{s,k} =
        \begin{bmatrix}
            \frac{{T_s}^3}{3 {T_k}^2} \\
            \frac{{T_s}^2}{{T_k}^2} \\
            \frac{2 {T_s}}{{T_k}^2} \\
        \end{bmatrix}
\end{equation}


\begin{listingtcb}{Maxima}
\begin{deflisting}
e: solve([(A.X0 + B.U)[2][1] = (X1)[2][1]], [dddx0]);
Dnew: coefmatrix([rhs(e[1])], list_matrix_entries(X0));
Enew: coefmatrix([rhs(e[1])], [dx1]);
e: subst(e, A.X0 + B.U);
Anew: coefmatrix(list_matrix_entries(e), list_matrix_entries(X0));
Bnew: coefmatrix(list_matrix_entries(e), [dx1]);
tex(Anew);
tex(Bnew);
tex(Dnew);
tex(Enew);
/* subsampling */
e: solve([(A.X0 + B.U)[2][1] = (X1)[2][1]], [dddx0]);
e: subst(e, As.X0 + Bs.U);
Asnew: coefmatrix(list_matrix_entries(e), list_matrix_entries(X0));
Bsnew: coefmatrix(list_matrix_entries(e), [dx1]);
tex(Asnew);
tex(Bsnew);
/* check */
subst([T_s = T_k], Asnew) - Anew;
subst([T_s = T_k], Bsnew) - Bnew;
\end{deflisting}
\end{listingtcb}



%%%%%%%%%%%%%%%%%%%%%%%%%%%%%%%%%%%%%%%%%%%%%%%%%%%%%%%%%%%%%%%%%%%%%%%%%%%%%%%%
\subsection{Controlled using position}

\begin{align}
    \V{x}_{k+1}
    &=
    \M{A}_k
    \V{x}_{k}
    +
    \M{B}_k
    {x}_{k+1}
    \\
    \dddot{x}_{k}
    &=
    \M{D}_k
    \V{x}_{k}
    +
    \M{E}_k
    {x}_{k+1}
\end{align}


%
\begin{equation}
    \M{A}_k =
        \begin{bmatrix}
            0                   &   0               & 0 \\
            - \frac{3}{T_k}     &  - 2              &  - \frac{T_k}{2} \\
            - \frac{6}{T_k^2}   &  - \frac{6}{T_k}  &  - 2 \\
        \end{bmatrix},
    \quad
    \M{B}_k =
        \begin{bmatrix}
            1 \\
            \frac{3}{T_k} \\
            \frac{6}{T_k^2} \\
        \end{bmatrix},
    \quad
    \M{D}_k =
        \begin{bmatrix}
            -\frac{6}{T_k^3} &   -\frac{6}{T_k^2} & -\frac{3}{T_k} \\
        \end{bmatrix},
    \quad
    \M{E}_k =
        \begin{bmatrix}
            \frac{6}{T_k^3} \\
        \end{bmatrix}
\end{equation}
%


For $T_s \in [0, T_k]$
%
\begin{equation}
    \M{A}_{s,k} =
        \begin{bmatrix}
            - \frac{{T_s}^3 - {T_k}^3}{{T_k}^3} &  - \frac{{T_s}^3 - {T_k}^2 {T_s}}{{T_k}^2}    &  - \frac{{T_s}^3 - {T_k}{T_s}^2}{2 {T_k}} \\
            - \frac{3 {T_s}^2}{{T_k}^3}         &  - \frac{3 {T_s}^2 - {T_k}^2}{{T_k}^2}        &  -  \frac{3 {T_s}^2 - 2 {T_k} {T_s}}{2 {T_k}} \\
            - \frac{6 {T_s}}{{T_k}^3}           &  - \frac{6 {T_s}}{{T_k}^2}                    &  - \frac{3 {T_s} - {T_k}}{{T_k}} \\
        \end{bmatrix}
    ,
    \quad
    \M{B}_{s,k} =
        \begin{bmatrix}
            \frac{{T_s}^3}{{T_k}^3} \\
            \frac{3 {T_s}^2}{{T_k}^3} \\
            \frac{6 {T_s}}{{T_k}^3} \\
        \end{bmatrix}
\end{equation}
%

\begin{listingtcb}{Maxima}
\begin{deflisting}
e: solve([(A.X0 + B.U)[1][1] = (X1)[1][1]], [dddx0]);
Dnew: coefmatrix([rhs(e[1])], list_matrix_entries(X0));
Enew: coefmatrix([rhs(e[1])], [x1]);
e: subst(e, A.X0 + B.U);
Anew: coefmatrix(list_matrix_entries(e), list_matrix_entries(X0));
Bnew: coefmatrix(list_matrix_entries(e), [x1]);
tex(Anew);
tex(Bnew);
tex(Dnew);
tex(Enew);
/* subsampling */
e: solve([(A.X0 + B.U)[1][1] = (X1)[1][1]], [dddx0]);
e: subst(e, As.X0 + Bs.U);
Asnew: coefmatrix(list_matrix_entries(e), list_matrix_entries(X0));
Bsnew: coefmatrix(list_matrix_entries(e), [x1]);
tex(Asnew);
tex(Bsnew);
/* check */
subst([T_s = T_k], Asnew) - Anew;
subst([T_s = T_k], Bsnew) - Bnew;
\end{deflisting}
\end{listingtcb}



%%%%%%%%%%%%%%%%%%%%%%%%%%%%%%%%%%%%%%%%%%%%%%%%%%%%%%%%%%%%%%%%%%%%%%%%%%%%%%%%
%%%%%%%%%%%%%%%%%%%%%%%%%%%%%%%%%%%%%%%%%%%%%%%%%%%%%%%%%%%%%%%%%%%%%%%%%%%%%%%%
%%%%%%%%%%%%%%%%%%%%%%%%%%%%%%%%%%%%%%%%%%%%%%%%%%%%%%%%%%%%%%%%%%%%%%%%%%%%%%%%
\section{Kronecker and Block Kronecker products}

The standard Kronecker product is defined as follows:
%
\begin{equation}
    \M{A} \kron \M{B}
    =
    \begin{bmatrix}
        A_{1,1} \M{B} & \dots  & A_{1,m} \M{B} \\
        \vdots        & \vdots & \vdots \\
        A_{n,1} \M{B} & \dots  & A_{n,m} \M{B} \\
    \end{bmatrix},
\end{equation}
%
where $\M{A} \in \RR^{n \times m}$ and $A_{i,j}$ denotes a scalar standing in
the $i$-th row and $j$-th column.


Derivations presented in this manual can often be expressed using a slightly
different operator:
%
\begin{equation}
    \M{A} \bkron \M{B}
    =
    \begin{bmatrix}
        \M{A} \kron \M{B}_{[1,1]}   & \dots  & \M{A} \kron \M{B}_{[1,m]} \\
        \vdots                      & \vdots & \vdots \\
        \M{A} \kron \M{B}_{[n,1]}   & \dots  & \M{A} \kron \M{B}_{[n,m]} \\
    \end{bmatrix},
\end{equation}
%
where $\M{B}_{[i,j]}$ denotes a block of matrix $\M{B}$. The operator $\bkron$
is called block Kronecker product and is usually defined in a more general
form, where matrix $\M{A}$ is also partitioned \cite{Tracy1972sn,
Hyland1989jmaa, Koning1991laa}. For the purpose of the present work
partitioning of $\M{A}$ and nonuniform partitioning of $\M{B}$ are not
necessary and therefore not considered.


Block Kronecker product has several valuable properties, notably:
%
\begin{equation}
    (\M{I} \bkron \M{A})
    (\M{I} \bkron \M{B})
    =
    \M{I}
    \bkron
    (\M{A} \M{B})
    ,
\end{equation}
%
which allows for more efficient computations. Another commonly used property is
the following:
%
\begin{equation}
    \diag{k = 1...N}
    {
        \M{I} \kron \M{A}_k
    }
    =
    \M{I}
    \bkron
    \diag{k = 1...N}
    {
        \M{A}_k
    }
\end{equation}
%


For example, consider a time-invariant system with the state transition matrix
$\M{A} = \M{I} \kron \barM{A}$ and control matrix $\M{B} = \M{I} \kron
\barM{B}$. Then the condensing matrices $\M{U}_{x}$ and $\M{U}_{u}$ can be
expressed as
%
\begin{equation}
    \M{U}_{x} =
        \M{I}
        \bkron
        \barM{U}_{x}
        =
        \M{I}
        \bkron
        \begin{bmatrix}
            \barM{A}    \\
            \barM{A}^2  \\
            \vdots   \\
            \barM{A}^N  \\
        \end{bmatrix}
        =
        \begin{bmatrix}
            \M{I} \kron \barM{A}    \\
            \M{I} \kron \barM{A}^2  \\
            \vdots   \\
            \M{I} \kron \barM{A}^N  \\
        \end{bmatrix}
\end{equation}
%
\begin{equation}
    \M{U}_{u} =
        \M{I}
        \bkron
        \barM{U}_{u}
        =
        \M{I}
        \bkron
        \begin{bmatrix}
            \barM{B}                    & \barM{0}                  & \dots & \barM{0} \\
            \barM{A} \barM{B}           & \barM{B}                  & \dots & \barM{0} \\
            \vdots                      & \vdots                    & \ddots& \vdots \\
            \barM{A}^{N-1} \barM{B}     & \barM{A}^{N-2} \barM{B}   & \dots & \barM{B} \\
        \end{bmatrix}
\end{equation}
%



\printbibliography[heading=bibintoc, title=Bibliography]
\end{refsection}

%%%%%%%%%%%%%%%%%%%%%%%%%%%%%%%%%%%%%%%%%%%%%%%%%%%%%%%%%%%%%%%%%%%%%%%%%%%%%%%%%%%%%%%%%%%%%%%%%%%%%%
%%%%%%%%%%%%%%%%%%%%%%%%%%%%%%%%%%%%%%%%%%%%%%%%%%%%%%%%%%%%%%%%%%%%%%%%%%%%%%%%%%%%%%%%%%%%%%%%%%%%%%
%%%%%%%%%%%%%%%%%%%%%%%%%%%%%%%%%%%%%%%%%%%%%%%%%%%%%%%%%%%%%%%%%%%%%%%%%%%%%%%%%%%%%%%%%%%%%%%%%%%%%%

% modules
\part{Walking Pattern Generators}\label{part.wpg}
\begin{refsection}


\chapter{Introduction}
\input{wpg_notation_define.tex}


%%%%%%%%%%%%%%%%%%%%%%%%%%%%%%%%%%%%%%%%%%%%%%%%%%%%%%%%%%%%%%%%%%%%%%%%%%%%%%%%%%%%%%%%%%%%%%%%%%%%%%
%%%%%%%%%%%%%%%%%%%%%%%%%%%%%%%%%%%%%%%%%%%%%%%%%%%%%%%%%%%%%%%%%%%%%%%%%%%%%%%%%%%%%%%%%%%%%%%%%%%%%%
%%%%%%%%%%%%%%%%%%%%%%%%%%%%%%%%%%%%%%%%%%%%%%%%%%%%%%%%%%%%%%%%%%%%%%%%%%%%%%%%%%%%%%%%%%%%%%%%%%%%%%
\section{Single point-mass models with coplanar contacts}


%%%%%%%%%%%%%%%%%%%%%%%%%%%%%%%%%%%%%%%%%%%%%%%%%%%%%%%%%%%%%%%%%%%%%%%%%%%%%%%%%%%%%%%%%%%%%%%%%%%%%%
\subsection{Common variables}

Position, velocity, acceleration, jerk of the \acs{CoM}:
\begin{align}
\cpos = \begin{bmatrix} c^x \\ c^y \end{bmatrix}
\quad&
\cPos = \begin{bmatrix} \cpos_1 \\ \vdots \\ \cpos_N \end{bmatrix}
\quad\quad\quad&
\cvel = \begin{bmatrix} \dot{c}^x \\ \dot{c}^y \end{bmatrix}
\quad&
\cVel = \begin{bmatrix} \cvel_1 \\ \vdots \\ \cvel_N \end{bmatrix}
\\
%
\cacc = \begin{bmatrix} \ddot{c}^x \\ \ddot{c}^y \end{bmatrix}
\quad&
\cAcc = \begin{bmatrix} \cacc_1 \\ \vdots \\ \cacc_N \end{bmatrix}
\quad\quad\quad&
\cjerk = \begin{bmatrix} \dddot{c}^x \\ \dddot{c}^y \end{bmatrix}
\quad&
\cJerk = \begin{bmatrix} \cjerk_0 \\ \vdots \\ \cjerk_{N-1} \end{bmatrix}
\end{align}


%%%%%%%%%%%%%%%%%%%%%%%%%%%%%%%%%%%%%%%%%%%%%%%%%%%%%%%%%%%%%%%%%%%%%%%%%%%%%%%%%%%%%%%%%%%%%%%%%%%%%%
\subsection{Discrete-time systems based on triple integrator}

The state of the \acs{CoM}:
\begin{equation}
\cstate = \begin{bmatrix} \cstate^x \\ \cstate^y \end{bmatrix} =
          \begin{bmatrix} c^x \\ \dot{c}^x \\ \ddot{c}^x \\ c^y \\ \dot{c}^y \\ \ddot{c}^y \end{bmatrix}
\quad\quad
\cState = \begin{bmatrix} \cstate_1 \\ \vdots \\ \cstate_N \end{bmatrix}\\
\end{equation}


Selection matrices:
\begin{equation}
    \M{I}_{p} =
    \begin{bmatrix}
        1 & 0 & 0 & 0 & 0 & 0 \\
        0 & 0 & 0 & 1 & 0 & 0
    \end{bmatrix}
    \quad
    \M{I}_{v} =
    \begin{bmatrix}
        0 & 1 & 0 & 0 & 0 & 0 \\
        0 & 0 & 0 & 0 & 1 & 0
    \end{bmatrix}
    \quad
    \M{I}_{a} =
    \begin{bmatrix}
        0 & 0 & 1 & 0 & 0 & 0 \\
        0 & 0 & 0 & 0 & 0 & 1
    \end{bmatrix}
\end{equation}
\begin{equation}
    \M{I}_{pv} =
    \begin{bmatrix}
        1 & 0 & 0 & 0 & 0 & 0 \\
        0 & 1 & 0 & 0 & 0 & 0 \\
        0 & 0 & 0 & 1 & 0 & 0 \\
        0 & 0 & 0 & 0 & 1 & 0 \\
    \end{bmatrix}
\end{equation}



%%%%%%%%%%%%%%%%%%%%%%%%%%%%%%%%%%%%%%%%%%%%%%%%%%%%%%%%%%%%%%%%%%%%%%%%%%%%%%%%%%%%%%%%%%%%%%%%%%%%%%
\subsubsection{Piece-wise constant jerk}

The dynamic system:
\begin{equation}
\V{\dot{x}} = \M{A} \V{x} + \M{B} \V{u}
\label{eq: dynamic system}
\end{equation}

can be defined as:
\begin{equation}
\begin{split}
  \V{x} = \cstate \\
  \V{u} = \cjerk
\end{split}
\end{equation}

resulting in:
\begin{equation}
\M{A}_k =
\begin{bmatrix}
    0 & 1 & 0   & 0 & 0 & 0\\
    0 & 0 & 1   & 0 & 0 & 0\\
    0 & 0 & 0   & 0 & 0 & 0\\
    0 & 0 & 0   & 0 & 1 & 0\\
    0 & 0 & 0   & 0 & 0 & 1\\
    0 & 0 & 0   & 0 & 0 & 0\\
\end{bmatrix}
\quad
\M{B}_k =
\begin{bmatrix}
    0         & 0\\
    0         & 0\\
    1  & 0\\
    0         & 0 \\
    0         & 0 \\
    0         & 1       \\
\end{bmatrix}
\end{equation}

After discretization, the model is now:
\begin{equation}
\begin{split}
    \cstate_{k+1} =& \M{A}_k \cstate_k + \M{B}_k \cjerk_k\\
    \cop_{k+1} =& \M{D}_{k} \cstate_{k+1}
\end{split}
\end{equation}

\begin{equation}
\M{A}_k =
\begin{bmatrix}
    1       & T_k   & T_k^2/2   & 0 & 0 & 0\\
    0       & 1     & T_k       & 0 & 0 & 0\\
    0       & 0     & 1         & 0 & 0 & 0\\
    0 & 0 & 0                   & 1       & T_k   & T_k^2/2   \\
    0 & 0 & 0                   & 0       & 1     & T_k       \\
    0 & 0 & 0                   & 0       & 0     & 1         \\
\end{bmatrix}
\quad
\M{B}_k =
\begin{bmatrix}
    T_k^3/6 & 0\\
    T_k^2/2 & 0\\
    T       & 0\\
    0       & T_k^3/6 \\
    0       & T_k^2/2 \\
    0       & T       \\
\end{bmatrix}
\end{equation}

\begin{equation}
\M{D}_{k} =
\begin{bmatrix}
    1       & 0     & -\frac{1}{\omega_k^2} & 0 & 0 & 0\\
    0 & 0 & 0                               & 1       & 0     & -\frac{1}{\omega_k^2}  \\
\end{bmatrix}
\quad
\omega_k = \sqrt{\frac{g}{c^z_k}}
\end{equation}


%%%%%%%%%%%%%%%%%%%%%%%%%%%%%%%%%%%%%%%%%%%%%%%%%%%%%%%%%%%%%%%%%%%%%%%%%%%%%%%%%%%%%%%%%%%%%%%%%%%%%%
\subsubsection{Piece-wise constant CoP velocity}

The same dynamic system of Eq.(\ref{eq: dynamic system}) can be defined as:
\begin{equation}
\begin{split}
  \V{x} = \cstate \\
  \V{u} = \dcop
\end{split}
\end{equation}

resulting in:
\begin{equation}
\M{A}_k =
\begin{bmatrix}
    0 & 1 & 0          & 0 & 0 & 0\\
    0 & 0 & 1          & 0 & 0 & 0\\
    0 & \omega^2 & 0  & 0 & 0 & 0\\
    0 & 0 & 0          & 0 & 1 & 0\\
    0 & 0 & 0          & 0 & 0 & 1\\
    0 & 0 & 0          & 0 & \omega^2 & 0\\
\end{bmatrix}
\quad
\M{B}_k =
\begin{bmatrix}
    0         & 0\\
    0         & 0\\
    -\omega^2 & 0\\
    0         & 0 \\
    0         & 0 \\
    0         & -\omega^2       \\
\end{bmatrix}
\end{equation}

Discretization in Maxima (see \cite{wiki2017discretization}):
\begin{listingtcb}{Maxima}
\begin{deflisting}
load("diag");
A: matrix([0, 1, 0], [0, 0, 1], [0, w^2, 0]);
B: matrix([0],[0],[-w^2]);
Ad: mat_function(exp, A*T);
Bd: expand(integrate(mat_function(exp, A*t), t, 0, T).B);
\end{deflisting}
\end{listingtcb}

The resulting model is:
\begin{equation}
\begin{split}
    \cstate_{k+1} =& \M{A}_k \cstate_k + \M{B}_k \dcop_k\\
    \cop_{k+1} =& \M{D}_{k} \cstate_{k+1}
\end{split}
\end{equation}

\begin{equation}
    \M{A}_k =
    \begin{bmatrix}
        1       & \frac{\sinh(T_k\omega_k)}{\omega_k} & \frac{\cosh(T_k\omega_k) - 1}{\omega_k^2} & 0 & 0 & 0\\
        0       & \cosh(T_k\omega_k)                & \frac{\sinh(T_k\omega_k)}{\omega_k}       & 0 & 0 & 0\\
        0       & \omega_k\sinh(T_k\omega_k)          & \cosh(T_k\omega_k)                      & 0 & 0 & 0\\
        0 & 0 & 0                   & 1       & \frac{\sinh(T_k\omega_k)}{\omega_k} & \frac{\cosh(T_k\omega_k) - 1}{\omega_k^2} \\
        0 & 0 & 0                   & 0       & \cosh(T_k\omega_k)                & \frac{\sinh(T_k\omega_k)}{\omega_k}       \\
        0 & 0 & 0                   & 0       & \omega_k\sinh(T_k\omega_k)          & \cosh(T_k\omega_k)                      \\
    \end{bmatrix}
\end{equation}
\begin{equation}
    \M{B}_k =
    \begin{bmatrix}
        -\frac{\sinh(T_k\omega_k)}{\omega_k} + T_k  & 0\\
        -\cosh(T_k\omega_k) + 1                     & 0\\
        -\omega \sinh(T_k\omega_k)                  & 0\\
        0       & -\frac{\sinh(T_k\omega_k)}{\omega_k} + T_k \\
        0       & -\cosh(T_k\omega_k) + 1                    \\
        0       & -\omega \sinh(T_k\omega_k)                 \\
    \end{bmatrix}
\end{equation}

The state of the system and the output matrix are the same as above.

%%%%%%%%%%%%%%%%%%%%%%%%%%%%%%%%%%%%%%%%%%%%%%%%%%%%%%%%%%%%%%%%%%%%%%%%%%%%%%%%%%%%%%%%%%%%%%%%%%%%%%
\subsubsection{Piece-wise constant CoP velocity (control is the CoP position)}
Let the system controlled by piece-wise constant \acs{CoP} velocity be defined as
%
\begin{equation}
\begin{split}
    \cstate_{k+1} =& \M{\tilde{A}}_k \cstate_k + \M{\tilde{B}}_k \dcop_k\\
    \cop_{k+1} =& \M{\tilde{D}}_{k} \cstate_{k+1},
\end{split}
\end{equation}
%
where matrices $\M{\tilde{A}}_k, \M{\tilde{B}}_k, \M{\tilde{D}}_{k}$ are
defined as above.


We can express $\cop_{k+1}$ using $\cstate_k$ and $\dcop_k$ as
%
\begin{equation}
    \cop_{k+1} = \M{\tilde{D}}_{k} \cstate_k + T_k \dcop_k.
\end{equation}
%
Hence we express the \acs{CoP} velocity through \acs{CoM} and \acs{CoP} positions
%
\begin{equation}
    \dcop_k = \frac{1}{T_k} \left(\cop_{k+1} - \M{\tilde{D}}_{k} \cstate_k \right),
\end{equation}
%
and substitute it back into our model:
%
\begin{equation}
\begin{split}
\cstate_{k+1} =
    \M{\tilde{A}}_k \cstate_k
    +
    \frac{1}{T_k} \M{\tilde{B}}_k \cop_{k+1}
    -
    \frac{1}{T_k} \M{\tilde{B}}_k \M{\tilde{D}}_{k} \cstate_k
    =
    \left( \M{\tilde{A}}_k - \frac{1}{T_k} \M{\tilde{B}}_k \M{\tilde{D}}_{k} \right) \cstate_k
    +
    \frac{1}{T_k} \M{\tilde{B}}_k \cop_{k+1}
\end{split}
\end{equation}
%
Thus we obtain a new model
%
\begin{align}
    \cstate_{k+1} =& \M{A} \cstate_k + \M{B} \cop_{k+1}\\
    \dcop_k =& \M{D}_k \cstate_k + \M{E}_k \cop_{k+1},
\end{align}
%
where
%
\begin{align}
    \M{A}_k &= \left( \M{\tilde{A}}_k - \frac{1}{T_k} \M{\tilde{B}}_k \M{\tilde{D}}_k \right)
    &
    \M{B}_k &= \frac{1}{T_k} \M{\tilde{B}}_k \\
    \M{D}_k &= - \frac{1}{T_k} \M{\tilde{D}}_k
    &
    \M{E}_k &= \diag{2}{\frac{1}{T_k}}
\end{align}
%


\begin{equation}
    \M{A}_k =
    \begin{bmatrix}
        {{\operatorname{sh}}\over{\omega_k\,T_k}}   &   {{\operatorname{sh}}\over{\omega_k}}    &   -{{\operatorname{sh}-\omega_k\,T_k\,\operatorname{ch}}\over{\omega_k^3\,T_k}}   & 0 & 0 & 0\\
        {{\operatorname{ch}-1}\over{T_k}}           &   \operatorname{ch}                       &   {{\omega_k\,T_k\,\operatorname{sh}-\operatorname{ch}+1}\over{\omega_k^2\,T_k}}  & 0 & 0 & 0\\
        {{\omega_k\,\operatorname{sh}}\over{T_k}}   &   \omega_k\,\operatorname{sh}             &   -{{\operatorname{sh}-\omega_k\,T_k\, \operatorname{ch}}\over{\omega_k\,T_k}}     & 0 & 0 & 0\\
        0 & 0 & 0                   & {{\operatorname{sh}}\over{\omega_k\,T_k}}   &   {{\operatorname{sh}}\over{\omega_k}}   &   -{{\operatorname{sh}-\omega_k\,T_k\,\operatorname{ch}}\over{\omega_k^3\,T_k}}\\
        0 & 0 & 0                   & {{\operatorname{ch}-1}\over{T_k}}           &   \operatorname{ch}                       &   {{\omega_k\,T_k\,\operatorname{sh}-\operatorname{ch}+1}\over{\omega_k^2\,T_k}}\\
        0 & 0 & 0                   & {{\omega_k\,\operatorname{sh}}\over{T_k}}   &   \omega_k\,\operatorname{sh}             &   -{{\operatorname{sh}-\omega_k\,T_k\, \operatorname{ch}}\over{\omega_k\,T_k}}\\
    \end{bmatrix},
\end{equation}
%
\begin{equation}
    \M{B}_k
    =
    \begin{bmatrix}
        -{{\operatorname{sh}-\omega_k\,T_k}\over{\omega_k\,T_k}} & 0\\
        -{{\operatorname{ch}-1}\over{T_k}}                       & 0\\
        -{{\omega_k\,\operatorname{sh}}\over{T_k}}               & 0\\
        0       & -{{\operatorname{sh}-\omega_k\,T_k}\over{\omega_k\,T_k}}\\
        0       & -{{\operatorname{ch}-1}\over{T_k}}\\
        0       & -{{\omega_k\,\operatorname{sh}}\over{T_k}}\\
    \end{bmatrix},
\end{equation}
%
where
%
\begin{equation}
    \begin{split}
        \operatorname{sh} &= \sinh (\omega_k\,T_k)\\
        \operatorname{ch} &= \cosh (\omega_k\,T_k)\\
    \end{split}
\end{equation}


Obtaining the matrices in Maxima:

\begin{listingtcb}{Maxima}
\begin{deflisting}
load("diag");
A: matrix([0, 1, 0], [0, 0, 1], [0, w^2, 0]);
B: matrix([0],[0],[-w^2]);
Ad: mat_function(exp, A*T);
Bd: expand(integrate(mat_function(exp, A*t), t, 0, T).B);
D: matrix([1, 0, -1/w^2]);

Anew: (Ad - Bd.D/T);
Bnew: Bd/T;
Dnew: -D/T;
Enew: matrix([1/T, 0], [0, 1/T]);

Anew: hypsimp(w*T, Anew);
Bnew: hypsimp(w*T, Bnew);

tex(Anew);
tex(Bnew);
\end{deflisting}
\end{listingtcb}

\let\cpos\THISCOMMANDSHOULDNEVERBEDEFINED
\let\cvel\THISCOMMANDSHOULDNEVERBEDEFINED
\let\cacc\THISCOMMANDSHOULDNEVERBEDEFINED
\let\cjerk\THISCOMMANDSHOULDNEVERBEDEFINED

\let\cPos\THISCOMMANDSHOULDNEVERBEDEFINED
\let\cVel\THISCOMMANDSHOULDNEVERBEDEFINED
\let\cAcc\THISCOMMANDSHOULDNEVERBEDEFINED
\let\cJerk\THISCOMMANDSHOULDNEVERBEDEFINED

\let\cstate\THISCOMMANDSHOULDNEVERBEDEFINED
\let\cState\THISCOMMANDSHOULDNEVERBEDEFINED

\let\cop\THISCOMMANDSHOULDNEVERBEDEFINED
\let\dcop\THISCOMMANDSHOULDNEVERBEDEFINED
\let\CoP\THISCOMMANDSHOULDNEVERBEDEFINED
\let\dCoP\THISCOMMANDSHOULDNEVERBEDEFINED

\let\fph\THISCOMMANDSHOULDNEVERBEDEFINED
\let\FPh\THISCOMMANDSHOULDNEVERBEDEFINED

\let\fp\THISCOMMANDSHOULDNEVERBEDEFINED
\let\FP\THISCOMMANDSHOULDNEVERBEDEFINED

\let\fd\THISCOMMANDSHOULDNEVERBEDEFINED
\let\FD\THISCOMMANDSHOULDNEVERBEDEFINED

\let\cp\THISCOMMANDSHOULDNEVERBEDEFINED



\subimport*{../../modules/wpg04/doc/}{wpg04.tex}

\printbibliography[heading=bibintoc, title=Bibliography]
\end{refsection}

\newcommand{\cbase}[1][{}]{\V{c}_{s#1}}
\newcommand{\cbody}[1][{}]{\V{c}_{d#1}}
\newcommand{\cbasestate}[1][{}]{\hatV{c}_{s#1}}
\newcommand{\cbodystate}[1][{}]{\hatV{c}_{d#1}}

\newcommand{\cvbase}[1][{}]{\dotV{c}_{s#1}}
\newcommand{\cabase}[1][{}]{\ddotV{c}_{s#1}}
\newcommand{\cjbase}[1][{}]{\dddotV{c}_{s#1}}

\newcommand{\cvbody}[1][{}]{\dotV{c}_{d#1}}
\newcommand{\cabody}[1][{}]{\ddotV{c}_{d#1}}
\newcommand{\cjbody}[1][{}]{\dddotV{c}_{d#1}}

\newcommand{\cBase}{\V{C}_s}
\newcommand{\cvBase}{\dotV{C}_s}

\newcommand{\cop}[1][{}]{\V[#1]{p}}
\newcommand{\cstate}{\hatV{c}}
\newcommand{\cjerk}{\dddotV{c}}
\newcommand{\cJerk}{\dddotV{C}}
\newcommand{\cState}{\hatV{C}}


\chapter{Inverse kinematics controller for Pepper}


%%%%%%%%%%%%%%%%%%%%%%%%%%%%%%%%%%%%%%%%%%%%%%%%%%%%%%%%%%%%%%%%%%%%%%%%%%%%%%%%%%%%%%%%%%%%%%%%%%%%%%
%%%%%%%%%%%%%%%%%%%%%%%%%%%%%%%%%%%%%%%%%%%%%%%%%%%%%%%%%%%%%%%%%%%%%%%%%%%%%%%%%%%%%%%%%%%%%%%%%%%%%%
%%%%%%%%%%%%%%%%%%%%%%%%%%%%%%%%%%%%%%%%%%%%%%%%%%%%%%%%%%%%%%%%%%%%%%%%%%%%%%%%%%%%%%%%%%%%%%%%%%%%%%
\section{Computation of wheel velocities}

Motion of the base in the inverse kinematics controller is represented with
translational and angular velocities. In order to execute a motion it is
necessary to map the corresponding velocities to wheel velocities, which are
the inputs of the wheel controllers of Pepper.


We assume that the $z$ axis of the global frame is normal to the contact
surface, and there is a frame $\FRAME{s}$ fixed to the base so that its $x$ and
$y$ axes span the contact surface and its origin is the center of rotation.
Then we can represent rotation in the $x$-$y$ plane from the global frame to
the base frame with $\M[][s]{R} \in \RR^{2 \CROSS 2}$.


Let $\V{v}_{s} = (\cvbase^{\MT{xy}}, {w}_{s}^{z})$ be the three-dimensional
vector of base velocity composed of translational velocity in $x$-$y$ plane and
angular velocity about axis $z$. In the following we derive matrix $\M{T}$ such
that $\V{v}_{w} = \M{T} \V[s]{v}_{s}$, where $\V[s]{v}_{s} = (\M[][s]{R}
\cvbase^{\MT{xy}}, {w}_{s}^{z})$ and $\V{v}_{w}$ is a three-dimensional vector
composed of scalar angular velocities of the wheels.


In order to compute $\M{T}$ we need to know the following parameters
%
\begin{longtable}[l]{@{\extracolsep{0pt}}l @{\extracolsep{3pt}}l p{14cm}}
    $\V[s]{d}_i$    &   $\in \RR^{3}$           &   distance to the ground contact point of
                                                    the $i$-th wheel in the base frame
                                                    (${d}_i^z = 0$)\\
    $r_i$           &   $\in \RR$               &   radius of the $i$-th wheel\\
    $\theta_i$      &   $\in \RR$               &   angle about $z$ axis representing orientation
                                                    of the $i$-th wheel in the base frame\\
\end{longtable}
%
Also, by definition, the translational velocity $\V{v}$ at distance $\V{r}$
from the center of rotation is computed as $\V{v} = \V{w} \CROSS \V{r}$, which
corresponds to scalar relation $v = w r$ when $\V{w} \bot \V{r}$.


Translational velocity at the wheel contact point in the base frame can be
computed as
%
\begin{equation}
    \V[s]{v}_i
    =
    \begin{bmatrix}
        \M[][s]{R} \cvbase^{\MT{xy}}\\
        0
    \end{bmatrix}
    +
    \begin{bmatrix}
        0\\

        0\\
        {w}_{s}^{z}
    \end{bmatrix}
    \CROSS
    \V[s]{d}_i
    =
    \begin{bmatrix}
        \begin{matrix}
            1 & 0\\
            0 & 1\\
            0 & 0
        \end{matrix}
        &
        -
        \V[s]{d}_i
        \CROSS
        \begin{bmatrix}
            0\\
            0\\
            1
        \end{bmatrix}
    \end{bmatrix}
    \begin{bmatrix}
        \M[][s]{R} \cvbase^{\MT{xy}}\\
        {w}_{s}^{z}
    \end{bmatrix}.
\end{equation}
%
We have to project $\V[s]{v}_i$ on the normal of this axis $\V[s]{n}_i$, which
is equal to the negative second column of rotation matrix computed based on
$\theta_i$. Hence, the velocity of the $i$-th wheel is
%
\begin{equation}
    {v}_i
    =
    {w}_i
    {r}_i
    =
    \T{\V[s]{n}_i}
    \begin{bmatrix}
        \begin{matrix}
            1 & 0\\
            0 & 1\\
            0 & 0
        \end{matrix}
        &
        -
        \V[s]{d}_i
        \CROSS
        \begin{bmatrix}
            0\\
            0\\
            1
        \end{bmatrix}
    \end{bmatrix}
    \V[s]{v}_{s}
    =
    \begin{bmatrix}
        \T{(\V[s]{n}_i^{xy})}
        &
        -
        \T{\V[s]{n}_i}
        \left(
            \V[s]{d}_i
            \CROSS
            \begin{bmatrix}
                0\\
                0\\
                1
            \end{bmatrix}
        \right)
    \end{bmatrix}
    \V[s]{v}_{s},
\end{equation}
%
and the $i$-th row of matrix $\M{T}$ can be deduced from equation
%
\begin{equation}
    {w}_i
    =
    \frac{1}{{r}_i}
    \begin{bmatrix}
        \T{(\V[s]{n}_i^{xy})}
        &
        -
        \T{\V[s]{n}_i}
        \left(
            \V[s]{d}_i
            \CROSS
            \begin{bmatrix}
                0\\
                0\\
                1
            \end{bmatrix}
        \right)
    \end{bmatrix}
    \V[s]{v}_{s}
\end{equation}
%



\let\cbase\THISCOMMANDSHOULDNEVERBEDEFINED
\let\cbody\THISCOMMANDSHOULDNEVERBEDEFINED
\let\cbasestate\THISCOMMANDSHOULDNEVERBEDEFINED
\let\cbodystate\THISCOMMANDSHOULDNEVERBEDEFINED

\let\cvbase\THISCOMMANDSHOULDNEVERBEDEFINED
\let\cabase\THISCOMMANDSHOULDNEVERBEDEFINED
\let\cjbase\THISCOMMANDSHOULDNEVERBEDEFINED

\let\cvbody\THISCOMMANDSHOULDNEVERBEDEFINED
\let\cabody\THISCOMMANDSHOULDNEVERBEDEFINED
\let\cjbody\THISCOMMANDSHOULDNEVERBEDEFINED

\let\cBase\THISCOMMANDSHOULDNEVERBEDEFINED
\let\cvBase\THISCOMMANDSHOULDNEVERBEDEFINED

\let\cop\THISCOMMANDSHOULDNEVERBEDEFINED
\let\cstate\THISCOMMANDSHOULDNEVERBEDEFINED
\let\cjerk\THISCOMMANDSHOULDNEVERBEDEFINED
\let\cJerk\THISCOMMANDSHOULDNEVERBEDEFINED
\let\cState\THISCOMMANDSHOULDNEVERBEDEFINED




%\appendix
\end{document}
